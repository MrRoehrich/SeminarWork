\section{Formulation} \label{sec:formul}
The phase-field model as a diffuse interface model does not introduce discontinuities into the solid but the fracture surface is approximated by a scalar valued field. Thus, the boundary between damaged and not damaged areas is smoothed. In the following parts, a model for brittle and ductile fracture as well as a second- and fourth-order model are presented. At first, the notation and formulation considering brittle fracture is outlined. Afterwards, differences in the formulation considering ductile fracture are shown. For the rest of this paper, following notation is made (see \figref{fig:bodies}\subrefnew{fig:body_1}). The arbitrary body $\Omega\subset\mathbb{R}^{d}$ ($d\in\{1,2,3\}$) has external boundary $\partial\Omega$ and evolving internal discontinuity boundary (fracture surface) $\Gamma$. The displacement field at a given point $\mathbf{x}$ and time $t$ is given by $\mathbf{u}\left(\mathbf{x},t\right)\in\mathbb{R}^{d}$. Dirichlet boundary conditions $\mathbf{u}\left(\mathbf{x},t\right)=\mathbf{g}\left(\mathbf{x},t\right)$ on $\partial\Omega_{\mathbf{g}}$ and Neumann boundary conditions $\mathbf{t}\left(\mathbf{x},t\right)=\mathbf{h}\left(\mathbf{x},t\right)$ on $\partial\Omega_{\mathbf{h}}$ are imposed with $\partial\Omega_{\mathbf{g}}\cup\partial\Omega_{\mathbf{h}}=\partial\Omega$. $\mathbf{t}\left(\mathbf{x},t\right)$ describes a given traction vector force.
\begin{figure}[ht!]
    \centering
    \begin{subfigure}[t]{0.4\textwidth}
        \centering
        \includegraphics[scale=0.3]{data/Body_1}
        \caption{}\label{fig:body_1}
    \end{subfigure}
    %
    \begin{subfigure}[t]{0.5\textwidth}
        \centering
        \includegraphics[scale=0.3]{data/Body_2}
        \caption{}\label{fig:body_2}
    \end{subfigure}
    \caption{\subrefnew{fig:body_1} Representation of a solid body $\Omega$ and internal discontinuitiy boundary $\Gamma$. \subrefnew{fig:body_2} Phase-field approximation of $\Gamma$. $c\left(\mathbf{x},t\right)$ describes the phase-field and $l_{0}$ is a parameter controlling the failure zone's width. \cite{01_PF_dyn_brittle}} \label{fig:bodies}
\end{figure}


\subsection{Griffith's theory of brittle fracture} \label{sec:formul_Griffith}
Considering small deformations and deformation gradients, the small strain tensor $\bm{\varepsilon}\left(\mathbf{x},t\right)$ is given by
\begin{equation}
	\bm{\varepsilon} = \nabla^{s}\mathbf{u}
\end{equation}
where $\left(\cdot\right)^{s}$ refers to the symmetric part. Considering isotropic linear elasticity, the undamaged elastic energy density can be expressed by
\begin{equation} \label{eq:psi_e}
	\Psi_ {e}\left(\bm{\varepsilon}\right) = \dfrac{1}{2}\lambda tr\left(\bm{\varepsilon}\right)^{2}+\mu\bm{\varepsilon}:\bm{\varepsilon}
\end{equation}
using the Lam\'{e} constants $\lambda$ and $\mu$ and $\left(\cdot\right):\left(\cdot\right)$ denoting the double contraction.

According to the energetic approaches to fracture, the critical fracture energy density $\mathcal{G}_{c}$ defines the energy being necessary to create a unit area of fracture surface. Since translation of cracks shall be forbidden and extension, branching and merging shall be allowed, there is an irreversibility condition stating $\Gamma\left(t\right)\subseteq\Gamma\left(t+\Delta t\right), \forall \Delta t>0$.

However, Griffith's fracture theory reaches its limits as soon as it is used to predict crack paths, nucleation of new cracks and complicated crack behaviours during kinking and branching. Thus, the problem is formulated in a variational sense which is shown in the following paragraphs. The phase-field approach can then be seen as a regularized version of this variational formulation. \citep{05_PF_ductile}

Newton's laws follow Hamilton's principle stating that the functional 
\begin{equation} \label{eq:fctal_Hamilton}
	J\left(q,\dot{q}\right)=\int\limits_{t_{0}}^{t_{1}}\mathcal{L}\left(q,\dot{q},t\right)\mathrm{d}t
\end{equation} reaches a stationary point. $\mathcal{L}\left(q,\dot{q},t\right)$ describes the so called Lagrangian and $q$ represents generalized coordinates. The motion of the mechanical system from $t_{0}$ to $t_{1}$ is then captured by this formulation \citep{01_B_LagrMech}. In the case presented here, the Lagrangian reads $\mathcal{L}\left(\mathbf{u},\dot{\mathbf{u}},\Gamma\right)=\Psi_{kin}\left(\mathbf{u}\right)-\Psi_{pot}\left(\mathbf{u},\Gamma\right)$. Inserting the introduced critical fracture energy density $\mathcal{G}_{c}$, the kinetic energy of the body and \eqref{eq:psi_e} leads to
\begin{equation} \label{eq:lagr}
	\mathcal{L}\left(\mathbf{u},\dot{\mathbf{u}},\Gamma\right) = \int\limits_{\Omega}\left(\frac{1}{2}\rho\dot{\mathbf{u}}\dot{\mathbf{u}}-\Psi_{e}\left(\bm{\varepsilon}\right)\right)\mathrm{d}\Omega - \int\limits_{\Gamma}\mathcal{G}_{c}\mathrm{d}\Gamma.
\end{equation}
The Euler-Lagrange Equation (ELE) is a differential equation whose solution satisfies the required equilibrium. Thus, this equation is also called the equation of motion \footnote{The term \textit{equation of motion} may be a bit misleading. It refers to the differential equation and not to its solution.}. A minimizer $q$ for \eqref{eq:lagr} satisfies the ELE
\begin{equation} \label{eq:ELE_O2}
	\dfrac{\partial\mathcal{L}}{\partial q}-\dfrac{\mathrm{d}}{\mathrm{d}t}\dfrac{\partial\mathcal{L}}{\partial\dot{q}}=0.
\end{equation}
For a given $\mathcal{L}=\mathcal{L}\left(q,\dot{q},\ddot{q}, t\right)$ this differential equation is changed to
\begin{equation} \label{eq:ELE:_O4}
	\dfrac{\partial\mathcal{L}}{\partial q}-\dfrac{\mathrm{d}}{\mathrm{d}t}\dfrac{\partial\mathcal{L}}{\partial\dot{q}}+\dfrac{\mathrm{d}^{2}}{\mathrm{d}t^{2}}\dfrac{\partial\mathcal{L}}{\partial\ddot{q}}=0
\end{equation}
\citep{01_B_LagrMech}. So as to circumvent problems of algorithmically tracking the propagating discontinuity $\Gamma$, the phase-field approach, which regularizes the just mentioned variational formulation, will be presented in the next chapters.

\subsection{Phase-field approximation} \label{sec:ph_approx}
As can be seen in \figref{fig:bodies}\subrefnew{fig:body_2} the fracture surface $\Gamma$ is approximated by a scalar valued field $c\left(\mathbf{x},t\right)$. This field is called the phase-field. Values of $c=1$ represent regions away from the crack (undamaged material) whereas $c=0$ symbolises the crack. Now, in \eqref{eq:lagr} the surface integral and thus, the need for tracking the crack can be eliminated. The approximation reads as follows:
\begin{equation} \label{eq:surf_int_approx}
	\int\limits_{\Gamma}\mathcal{G}_{c}\mathrm{d}\Gamma \approx \int\limits_{\Omega}\mathcal{G}_{c}\Gamma_{c,n}\mathrm{d}\Omega.
\end{equation}
Obviously, the surface integral can now be approximately calculated without knowing or tracking the fracture surface. \eqref{eq:surf_int_approx} represents the fracture surface energy. The quantity $\Gamma_{c,n}$ is called the crack density functional which depends on a parameter $l_{0}$, the phase-field $c\left(\mathbf{x},t\right)$ and its spatial derivatives up to order $\frac{n}{2}$ ($\frac{\partial c}{\partial \mathbf{x}},..,\frac{\partial^{\frac{n}{2}} c}{\partial \mathbf{x}^{\frac{n}{2}}}$) with $n$ being an even number. $l_{0}\in\mathbb{R}^{+}$ represents a parameter controlling the width of the approximation of the crack (see \figref{fig:bodies}\subrefnew{fig:body_2}). It could be seen as a numerical regularization parameter or as an material parameter. \citet{01_PF_dyn_brittle} showed that a critical stress level $\sigma_{c}$ depends on $l_{0}$. Thus, this parameter is here seen as a material property. For a more detailed discussion it is referred to the investigations of \citet{07_PF_l0}.

The notation $\Gamma_{c,n}$ already presages that $n$ determines the order of the approximation. \citet{02_PF_HO_brittle} presented a so called \textit{second-} and \textit{fourth-order phase-field theory}. For $n=2$, the crack density functional introduced by \citet{08_PF_Gammac2} is used where as for $n=4$, a new functional has been established:
\begin{align}
	\begin{aligned}   \label{eq:crack_dens_fctals}
		\Gamma_{c,2} &= \dfrac{1}{4l_{0}}\left[\left(c-1\right)^{2}+4l_{0}^{2}|\nabla c|^{2}\right], \\
		\Gamma_{c,4} &= \dfrac{1}{4l_{0}}\left[\left(c-1\right)^{2}+2l_{0}^{2}|\nabla c|^{2}+l_{0}^{4}\left(\Delta c\right)^{2}\right].
	\end{aligned}
\end{align}
These formulations have been analytically analysed. So as to keep things short, it is referred to the work by \citet{02_PF_HO_brittle} for more detailed information. Only one important aspect is mentioned here: ${\eqref{eq:crack_dens_fctals}}_{1}$ is well-posed variationally for all $c\in H^{1}\left(\Omega\right)$ and solutions will generally not show greater regularity. Thus, ${\eqref{eq:crack_dens_fctals}}_{2}$ has been introduced so as to provide higher regularity in the solutions.

\subsection{Energy approximation} \label{sec:energy_approx}
In the failure zone, there is a loss of material stiffness. So as to model this phenomenon, the elastic energy is split into contributions from tensile and compressive deformations\footnote{Originally, a parameter $k$ or $\eta<<1$ has been introduced by \citet{09_PF_k} into this equation so as to avoid ill-posedness. However, \citet{01_PF_dyn_brittle} found out that there is no necessity of setting $k>0$. Thus, the derivation of the governing evolution equations all include that $k=0$.}
\begin{equation} \label{eq:el_energy}
	\Psi_{e}\left(\bm{\varepsilon},c\right)=g\left(c\right) \Psi_{e}^{+}\left(\bm{\varepsilon}\right)+\Psi_{e}^{-}\left(\bm{\varepsilon}\right)
\end{equation}
with the so called degradation function $g\left(c\right)=c^{2}$. At a later point in this work, there will be more information on this function. This splitting is achieved with the help of spectral decomposition of the strain:
\begin{align} \label{eq:spectr_decomp}
	\begin{aligned}
		\bm{\varepsilon} = \mathbf{P}\bm{\Lambda}\mathbf{P}^{T} \quad \Rightarrow \quad \bm{\varepsilon}^{+} = \mathbf{P}\bm{\Lambda}^{+}\mathbf{P}^{T}, \text{ } \bm{\varepsilon}^{-} = \mathbf{P}\bm{\Lambda}^{-}\mathbf{P}^{T}, \\
		\bm{\Lambda}^{+}=diag\left(\left<\lambda_{1}\right>,\left<\lambda_{2}\right>,\left<\lambda_{3}\right>\right), \quad \left<x\right>=\begin{cases}x, &x>0 \\ 0, & x\leq0\end{cases}.
	\end{aligned}
\end{align}
$\bm{\Lambda}^{-}$ is analogously                                                                                                                                                                                                                                                                                                                                                                                                                                                                                                          defined. $\lambda_{i}\in\sigma\left(\bm{\varepsilon}\right),i\in\{1,2,3\}$ denote the eigenvalues of the strain tensor. Plugging \eqref{eq:spectr_decomp} into \eqref{eq:psi_e} leads to the energy contributions from tensile and compressive deformations:
\begin{align} \label{eq:psi_e+-}
	\begin{aligned}
		\Psi_{e}^{+}\left(\bm{\varepsilon}\right) &= \dfrac{1}{2}\lambda\left<tr\left(\bm{\varepsilon}\right)\right>^{2}+\mu tr\left[\left(\bm{\varepsilon}^{+}\right)^{2}\right], \\
		\Psi_{e}^{-}\left(\bm{\varepsilon}\right) &= \dfrac{1}{2}\lambda\left(tr\left(\bm{\varepsilon}\right)-\left<tr\left(\bm{\varepsilon}\right)\right>\right)^{2}+\mu tr\left[\left(\bm{\varepsilon}-\bm{\varepsilon}^{+}\right)^{2}\right].
	\end{aligned}
\end{align}
The assumption here is that the sign of the principal strains determine tensile and compressive contributions. Putting \eqref{eq:el_energy} and \eqref{eq:surf_int_approx} together leads to the Helmholtz free energy given by
\begin{equation} \label{eq:Helmholtz}
	\Psi_{n}=c^{2}\Psi_{e}^{+}+\Psi_{e}^{-}+\mathcal{G}_{c}\Gamma_{c,n}.
\end{equation}

\subsection{Strong form} \label{sec:strong_form}
The governing equations will include one for enforcing stress equilibrium and the other will govern the evolution of the phase-field.

Assuming given body forces $\mathbf{b}$ and traction vector forces $\mathbf{t}=\bm{\sigma}\mathbf{n}$ with outward-pointing normal vector on $\partial\Omega$, stress equilibrium is enforced by
\begin{equation} \label{eq:stress_equil}
	 \left\{\begin{alignedat}{2}
\nabla\cdot\bm{\sigma}+\mathbf{b} &= \rho\ddot{\mathbf{u}} && \quad\text{on } \Omega\times\left(0,T\right) \\
		\mathbf{u} &= \mathbf{g} && \quad\text{on } \partial\Omega_{\mathbf{g}}\times\left(0,T\right) \\
		\bm{\sigma}\mathbf{n} &= \mathbf{h} && \quad\text{on } \partial\Omega_{\mathbf{h}}\times\left(0,T\right) \\
		\mathbf{u} &= \mathbf{u}_{0} && \quad\text{on } \Omega\times0 \\
		\dot{\mathbf{u}} &= \mathbf{v}_{0} && \quad\text{on } \Omega\times0.
  \end{alignedat}\right.
\end{equation}
$\eqref{eq:stress_equil}_{1}$ represents the local form of the linear momentum balance with $\bm{\sigma}=c^{2}\frac{\partial\Psi_{e}^{+}}{\partial\bm{\varepsilon}}+\frac{\partial\Psi_{e}^{-}}{\partial\bm{\varepsilon}}$.

As described in \secref{sec:formul_Griffith}, the Euler-Lagrange Equation can be used to find a minimizer of \eqref{eq:fctal_Hamilton}. Plugging \eqref{eq:crack_dens_fctals} and \eqref{eq:el_energy} into \eqref{eq:lagr} as well as using \eqref{eq:surf_int_approx} makes the use of the ELE possible. For this case, $q\hat{=}c$ and $\frac{\mathrm{d}}{\mathrm{d}t}\hat{=}\frac{\mathrm{d}}{\mathrm{d}\mathbf{x}}$. All this together leads to the governing equations for the evolution of the phase-field, namely $\eqref{eq:c2_equil}_{1}$ for the second- and $\eqref{eq:c4_equil}_{1}$ for the fourth-order phase-field theory. As can be seen in these equations, $\Psi_{e}^{+}$ has been replaced by the strain history field $\mathcal{H}$ enforcing the irreversibility condition $\Gamma\left(t\right)\subseteq\Gamma\left(t+\Delta t\right), \forall \Delta t>0$ in the strong form. This field satisfies the Kuhn-Tucker conditions for loading and unloading \cite{01_PF_dyn_brittle}:
\begin{equation} \label{eq:KuhnTucker}
	\Psi_{e}^{+}-\mathcal{H}\leq0, \quad \dot{\mathcal{H}}\geq0, \quad \dot{\mathcal{H}}\left(\Psi_{e}^{+}-\mathcal{H}\right)=0.
\end{equation}
It can also be used to model pre-existing cracks or geometrical features \cite{01_PF_dyn_brittle}.

 \hl{\text{Boundary-Conditions!}}
\begin{equation} \label{eq:c2_equil}
		 \left\{\begin{alignedat}{2}
\left(\frac{4l_{0}\mathcal{H}}{\mathcal{G}_{c}}+1\right)c - 4l_{0}^{2}\Delta c &= 1 && \quad\text{on } \Omega\times\left(0,T\right) \\
\hl{\nabla c\cdot\mathbf{n}} &= 0 && \quad \text{on } \partial\Omega\times\left(0,T\right) \\
\mathcal{H} &= \mathcal{H}_{0} && \quad \text{on } \Omega\times0  
\end{alignedat}\right.
\end{equation}
\begin{equation} \label{eq:c4_equil}
		 \left\{\begin{alignedat}{2}
\left(\frac{4l_{0}\mathcal{H}}{\mathcal{G}_{c}}+1\right)c - 2l_{0}^{2}\Delta c + l_{0}^{4}\Delta\left(\Delta c\right) &= 1 && \quad\text{on } \Omega\times\left(0,T\right) \\
\hl{\Delta c} &= 0 && \quad \text{on } \partial\Omega\times\left(0,T\right) \\
\hl{\nabla\left(l_{0}^{4}\Delta c-2l_{0}^{2}c\right)\cdot\mathbf{n}} &= 0 && \quad \text{on } \partial\Omega\times\left(0,T\right) \\
\mathcal{H} &= \mathcal{H}_{0} && \quad \text{on } \Omega\times0  
\end{alignedat}\right.
\end{equation}
The numerical approximation of the solution of \eqref{eq:c2_equil} and \eqref{eq:c4_equil} are outlined in \secref{sec:num_formul}. The last topic of the this section is the difference in modelling ductile fracture in contrast to brittle fracture which has been considered up to this point.

\subsection{Ductile fracture} \label{sec:ductile_frac}
So as to motivate this section, at first the differences between models of brittle and ductile fracture are examined. Up to this point of this paper, linear elasticity has been assumed. In this context, brittle fracture has been formulated in a variational way. By introducing the phase-field approximation, a regularized formulation of the variational one has been established. The corresponding stress-strain-curves and the procedure are illustrated in \figref{fig:elastic}.
\begin{table}[!ht]
	\begin{center}
	\begin{tabular}{|c||c|c|c|}
		\cline{2-4}
			\multicolumn{1}{c||}{}& Linear elasticity & \multicolumn{2}{c|}{Brittle fracture} \\
 		\hline\hline
			\rotatebox[origin=c]{90}{ Process} & \raisebox{-.5\height}{\includegraphics[scale=0.3]{data/elastic_1}} & \raisebox{-.5\height}{\includegraphics[scale=0.3]{data/elastic_2}} & \raisebox{-.5\height}{\includegraphics[scale=0.3]{data/elastic_3}} \\
		\hline
			\rotatebox[origin=c]{90}{\small{ Formulation }} & $E\left(\mathbf{u}\right)$ & \begin{tabular}[c]{@{}c@{}}Variational formulation\\of brittle fracture\\$E\left(\mathbf{u},\Gamma\right)$\end{tabular}  & \begin{tabular}[c]{@{}c@{}}Phase-field formulation\\($\hat{=}$regularized version)\\$E\left(\mathbf{u},c\right)$\end{tabular} \\
		\hline
	\end{tabular}
	\end{center}
\captionof{figure}{Brittle fracture: Process from linear elasticity over the variational formulation towards the phase-field approximation. \cite{06_PF_ductile}} \label{fig:elastic}
\end{table}
The variational formulation has been well established. Thus, the regularized version using the phase-field approximation could be found. As the dashed lines in \figref{fig:plastic} reveal, there is no variational formulation of ductile fracture found yet. Thus, the foundation of the regularized version is not as easy as for brittle fracture.

\begin{table}[!ht]
	\begin{center}
	\begin{tabular}{|c||c|c|c|}
		\cline{2-4}
			\multicolumn{1}{c||}{}& Plasticity & \multicolumn{2}{c|}{Ductile fracture} \\
 		\hline\hline
			\rotatebox[origin=c]{90}{ Process} & \raisebox{-.5\height}{\includegraphics[scale=0.3]{data/plastic_1}} & \raisebox{-.5\height}{\includegraphics[scale=0.3]{data/plastic_2}} & \raisebox{-.5\height}{\includegraphics[scale=0.3]{data/plastic_3}} \\
		\hline
			\rotatebox[origin=c]{90}{\small{ Formulation }} & $E\left(\bm{\varepsilon}^{e},\bm{\varepsilon}^{p},\alpha\right)$ & \begin{tabular}[c]{@{}c@{}}Variational formulation\\of ductile fracture\\$E\left(\bm{\varepsilon}^{e},\bm{\varepsilon}^{p},\alpha,\Gamma\right)$\end{tabular}  & \begin{tabular}[c]{@{}c@{}}Phase-field formulation\\($\hat{=}$regularized version)\\$E\left(\bm{\varepsilon}^{e},\bm{\varepsilon}^{p},\alpha,c\right)$\end{tabular} \\
		\hline
	\end{tabular}
	\end{center}
\captionof{figure}{Ductile fracture: Process from plasticity towards the phase-field approximation. The variational formulation has not been established yet. \cite{06_PF_ductile}} \label{fig:plastic}
\end{table}

\citet{06_PF_ductile} proposes a method similar to the one presented for brittle fracture. The major change lies in the Helmholtz free energy (see \eqref{eq:Helmholtz}) which is replaced by
\begin{equation} \label{eq:Helmholtz_ductile1}
	\Psi_{n} = g\left(c,p\right)\Psi_{e}^{+}\left(\bm{\varepsilon}\right)+\Psi_{e}^{-}\left(\bm{\varepsilon}\right)+\Psi_{p}\left(\alpha\right)+\mathcal{G}_{c}\Gamma_{c,n}
\end{equation}
with the new degradation function $g\left(c,p\right)=c^{2p}+\eta$ and $p=\frac{\epsilon_{eq}^{p}}{\epsilon_{eq,crit}^{p}}$, $\epsilon_{eq}^{p}\left(t\right)=\sqrt{\frac{2}{3}}\int\limits_{0}^{t}\sqrt{\dot{\bm{\varepsilon}}:\dot{\bm{\varepsilon}}}\mathrm{d}t$. The plastic energy density function considering linear isotropic hardening reads $\Psi_{p}\left(\alpha\right)=\sigma_{y}\alpha+\frac{1}{2}h\alpha^{2}$ with yield stress $\sigma_{y}$, hardening variable $\alpha$ and hardening modulus $h>0$. The idea in this new degradation function is that $c$ depends on $p$ so that the fracture process can be seen as the accumulation of ductile damage. $\Psi_{p}$ will take the dominating role over $\Psi_{e}^{+}$ and thus, a plasticity-driven fracture initiation is achieved. However, here a more abstract derivation of the governing equations will be presented which is based on the work of \citet{03_PF_ductile}.

In \secref{sec:formul_Griffith} small deformations and deformation gradients have been assumed. Since this does not hold true for ductile fracture (see \figref{fig:plastic}), another approach for the derivation of the strong form is chosen. As \citet{03_PF_ductile} have outlined, previous approaches on ductile fracture suffer in the fact that yield surface and hardening modulus are not effected by the phase-field's evolution. Thus, at some point plastic strains saturate and deformation is again dominated by recoverable elastic strains. The notation mentioned at the beginning of this section stays the same. The deformation gradient, which cannot be considered small anymore, now reads
\begin{equation} \label{eq:deform_grad}
	\mathbf{F}=\dfrac{\partial\varphi\left(\mathbf{X},t\right)}{\partial\mathbf{X}}
\end{equation}
with deformation map $\varphi:\left(\Omega_{0}\times\mathbb{R}\right)\rightarrow\mathbb{R}^{d}$, body $\Omega_{0}$ in the reference configuration  and $\mathbf{x}=\varphi\left(\mathbf{X},t\right)$. \eqref{eq:deform_grad} is now decomposed into the elastic and plastic deformations gradient:
\begin{equation} \label{eq:deform_grad_decomp}
	\mathbf{F}=\mathbf{F}^{e}\mathbf{F}^{p}.
\end{equation}
Compared to \eqref{eq:Helmholtz}, the new stored energy functional proposed by \citet{03_PF_ductile} reads
\begin{equation} \label{eq:fctal_ductile}
	\tilde{\Psi}\left(\mathbf{F},c,c_{,\mathbf{X}},\mathbf{F}^{p},\alpha\right) = \int\limits_{\Omega_{0}}\left[g\left(c\right)\Psi^{+}\left(\mathbf{F},\mathbf{F}^{p}\right)+\Psi^{-}\left(\mathbf{F},\mathbf{F}^{p}\right)+g_{p}\left(c\right)\Psi_{p}\left(\alpha\right)+\Gamma_{c,n}\right]\mathrm{d}\Omega_{0}
\end{equation}
with $\left(\cdot\right)_{\mathbf{X}}$ denoting the derivative with respect to $\mathbf{X}$.\footnote{Note that for ductile fracture in this work, $\nabla\left(\cdot\right)$ is replaced by the index notation $\left(\cdot\right)_{,\mathbf{X}}$ or $\left(\cdot\right)_{,\mathbf{x}}$, respectively, in order to distinguish between derivatives with respect to the reference or current configuration. For brittle fracture, everything has been derived within the current configuration.} As in the work by \citet{06_PF_ductile}, an effective plastic work contribution $\Psi_{p}$ depending on the internal hardening variable $\alpha$ has been added. The plastic degradation function $g_{p}\left(c\right)$ provides, analogously to $g\left(c\right)$, a mechanism involving crack growth driven by the development of plastic strains. For example, $g\left(c\right)$ leads to the neglection of plastic softening. In \citep{03_PF_ductile} it has been examined that $g\left(c\right)=c^{2}$ is not leading to a linear stress-strain-curve up to the point of critical stress. To overcome this problem, \citet{03_PF_ductile} propose following parametrized cubic degradation function:
\begin{equation} \label{eq:cubic_degr_fct}
	g\left(c\right)=m\left(c^{3}-c^{2}\right)+3c^{2}-2c^{3}
\end{equation}
with $m>0$ determining the slope of $g$ at $c=1$. It can be shown that for strains smaller than the critical strain for the quadratic degradation function the cubic degradation function leads to $\sigma=E\varepsilon$ as $m\rightarrow0$. Thus, $m=10^{-4}$ has been chosen in their numerical examples. The two major accomplishments using the cubic degradation function are at first, there is a nearly linear stress-strain behaviour up to the point of critical stress. Thus, linear elasticity is modelled more accurately. Secondly, it shows up that the critical stress is higher than for the quadratic degradation function (for $l_{0}$ fixed). Thus, a larger length scale parameter can be used so as to achieve the critical stress since $\sigma_{crit}^{cubic}\sim l_{0}^{-\frac{1}{2}}$. Thus, coarser meshes can be used in order to reduce computational effort.

The approach for the derivation of the governing equations is based on balance laws. To keep things short, not every single step of their derivation but more likely their results are presented here. Microforce balance laws are assumed to model the phase-field's evolution within a large deformation setting. The balance of linear momentum and angular momentum in the reference configuration lead to (also see \eqref{eq:stress_equil})
\begin{equation} \label{eq:lin_mom_ang_mom}
	\begin{aligned}
		Div\left(\mathbf{P}\right)+\mathbf{B}&=\rho_{0}\ddot{\mathbf{U}}, \\
		\bm{\sigma} &= \bm{\sigma}^{T}
	\end{aligned}
\end{equation}
with $1^{st}$ Piola-Kirchhoff stress tensor $\mathbf{P}$, density $\rho_{0}$ in the reference configuration and $Div\left(\cdot\right)$ denoting the divergence in the reference configuration. Analogously, assuming an internal microforce $\pi\left(\mathbf{x},t\right)\in\mathbb{R}$, an external microforce $l\left(\mathbf{x}.t\right)\in\mathbb{R}$ acting on the body and an external microforce $\lambda\left(\mathbf{x},t\right)=\bm{\xi}\cdot\mathbf{N}$ acting on the surface (with microforce traction vector $\bm{\xi}\left(\mathbf{x},t\right)\in\mathbb{R}^{d}$), the microforce balance leads to
\begin{equation} \label{eq:microforce_balance}
	Div\left(\bm{\xi}\right)+\pi+l=0.
\end{equation}
The energy balance with $e_{0}$ describing the internal energy per unit volume reads
\begin{equation} \label{eq:energy_balance}
	\int\limits_{\Omega_{0}}\dot{e}_{0}\mathrm{d}\Omega_{0}=\int\limits_{\Omega_{0}}\mathbf{P}:\dot{\mathbf{U}}\mathrm{d}\Omega_{0}+\int\limits_{\partial\Omega_{0}}\lambda\dot{c}\text{ }\mathrm{d}\partial\Omega_{0}+\int\limits_{\Omega_{0}}l\dot{c}\text{ }\mathrm{d}\Omega_{0},
\end{equation}
where $\dot{c}$ is the work conjugate of the microforces. The second law of thermodynamics leads to
\begin{equation}
	\Theta\dot{s}\geq0
\end{equation}
with entropy per unit volume $s$ and absolute temperature $\Theta$. Note that heat fluxes and heat sources are not considered here (isothermal and adiabatic process). Using the dissipation inequality $e_{0}=\Psi+\Theta  s$ (see Legendre transformation) this can be rewritten as
\begin{equation} \label{eq:entropy_balance}
	\dot{\Psi}\leq\dot{e}_{0}
\end{equation}
with Helmholtz free energy $\Psi$. Using \eqref{eq:energy_balance} and \eqref{eq:entropy_balance}, applying the divergence theorem, integrating by parts and using \eqref{eq:microforce_balance}, a thermodynamically consistent model is achieved as long as $\Psi$ fulfills \hl{wieso U. -> F.}
\begin{equation} \label{eq:thermodyn_cons}
		\int\limits_{\Omega_{0}}\dot{\Psi}\mathrm{d}\Omega_{0} \leq \int\limits_{\Omega_{0}}\left(\mathbf{P}:\dot{\mathbf{F}}+\bm{\xi}\cdot\dot{c}_{,\mathbf{X}}-\pi\dot{c}\right)\mathrm{d}\Omega_{0} = \int\limits_{\Omega_{0}}\left(\dfrac{1}{2}\mathbf{S}:\dot{\mathbf{C}}+\bm{\xi}\cdot\dot{c}_{,\mathbf{X}}\pi\dot{c}\right)\mathrm{d}\Omega_{0}.
\end{equation}
$\mathbf{S}$ denotes the $2^{nd}$ Piola-Kirchhoff stress tensor and $\mathbf{C}=\mathbf{F}^{T}\mathbf{F}$ the right Cauchy-Green tensor.

Assuming that $\Psi=\Psi\left(\mathbf{C},\mathbf{C}^{p},\mathbf{Q},c,c_{,\mathbf{X}},\dot{c}\right)$ with plastic right Cauchy-Green tensor $\mathbf{C}^{p}={\mathbf{F}^{p}}^{T}\mathbf{F}^{p}$ and set of internal plastic variables $\mathbf{Q}$, applying the chain rule in \eqref{eq:thermodyn_cons} and integrating over the reference domain leads to
\begin{equation} \label{eq:thermodyn_cons2}
	\begin{aligned}
	\int\limits_{\Omega_{0}}&\left(\dfrac{\partial\Psi}{\partial\mathbf{C}}:\dot{\mathbf{C}}+\dfrac{\partial\Psi}{\partial\mathbf{C}^{p}}:\dot{\mathbf{C}}^{p}+\dfrac{\partial\Psi}{\partial\mathbf{Q}}\cdot\dot{\mathbf{Q}}+\dfrac{\partial\Psi}{\partial c}\dot{c}+\dfrac{\partial\Psi}{\partial c_{,\mathbf{X}}}\cdot\dot{c}_{,\mathbf{X}}+\dfrac{\partial\Psi}{\partial\dot{c}}\ddot{c}\right)\mathrm{d}\Omega_{0} \\
	&\leq \int\limits_{\Omega_{0}}\left(\dfrac{1}{2}\mathbf{S}:\dot{\mathbf{C}}+\bm{\xi}\cdot\dot{c}_{,\mathbf{X}}-\pi\dot{c}\right)\mathrm{d}\Omega_{0}.
	\end{aligned}
\end{equation}
The plastic dissipation $\mathbb{D}^{p}=-\dfrac{\partial\Psi}{\partial\mathbf{C}^{p}}:\dot{\mathbf{C}}^{p}-\dfrac{\partial\Psi}{\partial\mathbf{Q}}\cdot\dot{\mathbf{Q}}$ can be set to zero for purely elastic deformations. Comparing the terms on the left hand and right hand side of \eqref{eq:thermodyn_cons2}, the following relations can be derived:
\begin{equation} \label{eq:relations_cons2}
	\begin{aligned}
		\dfrac{\partial\Psi}{\partial\dot{c}} &= 0, \\
		\dfrac{\partial\Psi}{\partial c_{,\mathbf{X}}} &= \bm{\xi}, \\
		2\dfrac{\partial\Psi}{\partial\mathbf{C}} &= \mathbf{S}, \\
		\left(\pi+\dfrac{\partial\Psi}{\partial c}\right)\dot{c} &\leq 0, \\
		\mathbb{D}^{p} &\geq 0.
	\end{aligned}
\end{equation}
$\eqref{eq:relations_cons2}_{4}$ can be rewritten as
\begin{equation} \label{eq:relation_cons_beta}
	\pi = -\dfrac{\partial\Psi}{\partial c}-\beta\dot{c}
\end{equation}
with function $\beta=\beta\left(\mathbf{C},\mathbf{C}^{p},c,c_{,\mathbf{X}}\right)\geq0$.\footnote{$\beta$ can be seen as a dissipation constant.} Inserting \eqref{eq:relations_cons2} and \eqref{eq:relation_cons_beta} into the balance laws $\eqref{eq:lin_mom_ang_mom}_{1}$ and \eqref{eq:microforce_balance} leads to following equations (using $\mathbf{P}=\mathbf{F}\mathbf{S}$):
\begin{equation} \label{eq:balance_laws}
	\begin{aligned}
		Div\left(2\mathbf{F}\dfrac{\partial\Psi}{\partial\mathbf{C}}\right)+\mathbf{B} &= \rho_{0}\ddot{\mathbf{U}}, \\
		Div\left(\dfrac{\partial\Psi}{\partial c_{,\mathbf{X}}}\right)+l-\dfrac{\partial\Psi}{\partial c} &= \beta\dot{c}.
	\end{aligned}
\end{equation}
Considering \eqref{eq:fctal_ductile}, the Helmholtz free energy for ductile fracture reads
\begin{equation} \label{eq:Helmholtz_ductile2}
	\Psi_{n}\left(\mathbf{C},\mathbf{C}^{p},\mathbf{Q},c,c_{,\mathbf{X}}\right) = g\left(c\right)\Psi^{+}\left(\mathbf{C},\mathbf{C}^{p}\right)+\Psi^{-}\left(\mathbf{C},\mathbf{C}^{p}\right)+g_{p}\left(c\right)\Psi_{p}\left(\mathbf{Q}\right)+\Gamma_{c,n}.
\end{equation}
For the fourth-order phase-field theory $\Psi$ also depends on the second-order spatial derivatives of $c$. Using this, the partial derivatives required in \eqref{eq:balance_laws} can be computed. Furthermore, setting $l=0$ and $\beta=0$ leads to the governing equations. The stress equilibrium is given by
\begin{equation} \label{eq:stress_equil_ductile}
	 \left\{\begin{alignedat}{2}
		Div\left(2\mathbf{F}\left(g\left(c\right)\dfrac{\partial\Psi^{+}}{\partial\mathbf{C}}+\dfrac{\partial\Psi^{-}}{\partial\mathbf{C}}\right)\right)+\mathbf{B} &= \rho_{0}\ddot{\mathbf{U}} && \quad\text{on } \Omega_{0}\times\left(0,T\right) \\
		\mathbf{U} &= \mathbf{G} && \quad\text{on } \partial\Omega_{0,\mathbf{G}}\times\left(0,T\right) \\
		\mathbf{T} &= \mathbf{H} && \quad\text{on } \partial\Omega_{0,\mathbf{H}}\times\left(0,T\right) \\
		\mathbf{U} &= \mathbf{U}_{0} && \quad\text{on } \Omega_{0}\times0 \\
		\dot{\mathbf{U}} &= \mathbf{V}_{0} && \quad\text{on } \Omega_{0}\times0
  \end{alignedat}\right.
\end{equation}
with Piola-Kirchhoff traction vector $\mathbf{T}=\mathbf{P}\cdot\mathbf{N}$. Considering the second-order phase-field theory and inserting the two partial derivatives $\frac{\partial\Psi}{\partial c_{,\mathbf{X}}}$ and $\frac{\partial\Psi}{\partial c}$ required in \eqref{eq:balance_laws} with the help of \eqref{eq:Helmholtz_ductile2} leads to following equation:
\begin{equation} \label{eq:c2_ph_evolution_tmp}
\dfrac{2l_{0}}{\mathcal{G}_{c}^{0}}\left(g'\left(c\right)\Psi^{+}+g_{p}'\left(c\right)\Psi_{p}\right) + c - 4l_{0}^{2}\Delta c = 1 \quad\text{on } \Omega\times\left(0,T\right).
\end{equation}
However, the irreversibility conditions outlined in \eqref{eq:KuhnTucker} have not been considered yet. Now, the history field is given by
\begin{equation} \label{eq:history_field_ductile}
	\mathcal{H}\left(\mathbf{C},\mathbf{C}^{p}\right) = \max\limits_{\bar{t}\leq t} \Psi^{+}\left(\mathbf{C}\left(\bar{t}\right),\mathbf{C}^{p}\left(\bar{t}\right)\right).
\end{equation}
Having $\left<\cdot\right>$ defined as in \eqref{eq:spectr_decomp}, the term $\Psi_{p}$ in \eqref{eq:c2_ph_evolution_tmp} is replaced by $\left<\Psi_{p}-\Psi_{0}\right>$. Actually, $\Psi_{p}$ is already assumed to be monotonically increasing so that no additional constraints are necessary in order to enforce irreversibility crack growth caused by plastic deformation. However, \citet{03_PF_ductile} introduce the plastic work threshold $\Psi_{0}$ so as to have more control over the contribution of plastic deformation to crack growth. Thus, if $\Psi_{p}<\Psi_{0}$, there will be no contribution of plastic deformation to crack growth. This point can now be controlled easily. Furthermore, \citet{03_PF_ductile} introduce parameters $\beta_{e}\in\left[0,1\right]$ and $\beta_{p}\left[0,1\right]$ which can be used in order to weight the contributions from elastic strain energy and plastic work to crack growth. These modifications, \eqref{eq:c2_ph_evolution_tmp} and \eqref{eq:history_field_ductile} lead to the governing equations for the evolution of the phase-field considering ductile fracture and the second-order phase-field theory:
\begin{equation} \label{eq:c2_equil_ductile}
	\left\{\begin{alignedat}{2}
		\dfrac{2l_{0}}{\mathcal{G}_{c}^{0}}\left(\beta_{e}g'\left(c\right)+\beta_{p}g_{p}'\left(c\right)\left<\Psi_{p}-\Psi_{0}\right>\right) + c - 4l_{0}^{2}\Delta c\ &= 1 && \quad\text{on } \Omega_{0}\times\left(0,T\right) \\
		\hl{c_{,\mathbf{X}}\cdot\mathbf{N}} &= 0 && \quad \text{on } \partial\Omega_{0}\times\left(0,T\right) \\
		\mathcal{H} &= \mathcal{H}_{0} && \quad \text{on } \Omega_{0}\times0.
	\end{alignedat}\right.
\end{equation}
There are several constitutive models so as to determine the energy density functions $\Psi^{+}$, $\Psi^{-}$ and $\Psi_{p}$. These have been well outlined by \citet{03_PF_ductile} and within this work, it is only referred to this work. \hl{initial condition right?! or u0 v0 sufficient? -> dissertation Borden}

Since $\eqref{eq:crack_dens_fctals}_{2}$ includes second order spatial derivatives of $c$ for the fourth-order phase-field theory, the relations \eqref{eq:relations_cons2} have to be adapted. For that, the Helmholtz free energy is assumed to take the form $\Psi=\Psi\left(\mathbf{C},\mathbf{C}^{p},\mathbf{Q},c,c_{,\mathbf{X}},c_{,ii},\dot{c}\right)$ so that \eqref{eq:thermodyn_cons2} changes to:
\begin{equation} \label{eq:thermodyn_cons4}
	\begin{aligned}
	\int\limits_{\Omega_{0}}&\left(\dfrac{\partial\Psi}{\partial\mathbf{C}}:\dot{\mathbf{C}}+\dfrac{\partial\Psi}{\partial\mathbf{C}^{p}}:\dot{\mathbf{C}}^{p}+\dfrac{\partial\Psi}{\partial\mathbf{Q}}\cdot\dot{\mathbf{Q}}+\dfrac{\partial\Psi}{\partial c}\dot{c}+\dfrac{\partial\Psi}{\partial c_{,\mathbf{X}}}\cdot\dot{c}_{,\mathbf{X}}+\dfrac{\partial\Psi}{\partial c_{,jj}}\dot{c}_{,ii}+\dfrac{\partial\Psi}{\partial\dot{c}}\ddot{c}\right)\mathrm{d}\Omega_{0} \\
	&\leq \int\limits_{\Omega_{0}}\left(\dfrac{1}{2}\mathbf{S}:\dot{\mathbf{C}}+\bm{\xi}\cdot\dot{c}_{,\mathbf{X}}-\pi\dot{c}\right)\mathrm{d}\Omega_{0}.
	\end{aligned}
\end{equation}
So as to compare the terms on the left- and right-hand side and thus, to involve the Laplacian of $c$, the divergence theorem is applied:
\begin{equation} \label{eq:div_theorem_Lapl_c}
	\begin{aligned}
		\int\limits_{\Omega_{0}}\dfrac{\partial\Psi}{\partial c_{,jj}}\dot{c}_{,ii}\mathrm{d}\Omega_{0} &= \int\limits_{\Omega_{0}}\left[\left(\dfrac{\partial\Psi}{\partial c_{,jj}}\dot{c}_{,i}\right)_{,i}-\left(\dfrac{\partial\Psi}{\partial c_{,jj}}\right)_{,i}\dot{c}_{,i}\right]\mathrm{d}\Omega_{0} \\
		&= \int\limits_{\partial\Omega_{0}}\dfrac{\partial\Psi}{\partial c_{,jj}}\dot{c}_{,i}N_{i}\mathrm{d}\partial\Omega_{0}-\int\limits_{\Omega_{0}}\left(\dfrac{\partial\Psi}{\partial c_{,jj}}\right)_{,i}\dot{c}_{,i}\mathrm{d}\Omega_{0}.
	\end{aligned}
\end{equation}
As a consequence, terms can be compared which leads to following new relations:
\begin{equation} \label{eq:relations_cons4}
	\begin{aligned}
		\dfrac{\partial\Psi}{\partial\dot{c}} &= 0, \\
		\dfrac{\partial\Psi}{\partial c_{,\mathbf{i}}} -\left(\dfrac{\partial\Psi}{\partial c_{,jj}}\right)_{,i} &= \xi_{,i}, \\
		\dfrac{\partial\Psi}{\partial c_{,jj}} &= 0 \quad \text{on } \partial\Omega_{0}, \\
		2\dfrac{\partial\Psi}{\partial\mathbf{C}} &= \mathbf{S}, \\
		\left(\pi+\dfrac{\partial\Psi}{\partial c}\right)\dot{c} &\leq 0, \\
		\mathbb{D}^{p} &\geq 0.
	\end{aligned}
\end{equation}
The function $\beta$ stays the same as in \eqref{eq:relation_cons_beta}. Inserting this function and the relations \eqref{eq:relations_cons4} into the microforce balance \eqref{eq:microforce_balance} leads to the equation
\begin{equation} \label{eq:balance_law_c4_ductile}
	\left(\dfrac{\partial\Psi}{\partial c_{,i}}\right)_{,i}-\left(\dfrac{\partial\Psi}{\partial c_{,jj}}\right)_{,ii}-\dfrac{\partial\Psi}{\partial c}-\beta\dot{c}+l=0.
\end{equation}
Inserting the Helmholtz free energy from \eqref{eq:Helmholtz_ductile2} for $n=4$ into \eqref{eq:balance_law_c4_ductile} leads to the governing equations for the phase-field's evolution considering the fourth-order phase-field theory:
\begin{equation} \label{eq:c4_equil_ductile}
	\left\{\begin{alignedat}{2}
		\dfrac{2l_{0}}{\mathcal{G}_{c}^{0}}\left(g'\left(c\right)\Psi^{+}+g_{p}'\left(c\right)\Psi_{p}\right) + c - 2l_{0}^{2}\Delta c +l_{0}^{4}\Delta\left(\Delta c\right) &= 1 && \quad\text{on } \Omega\times\left(0,T\right) \\
		\hl{\Delta c} &= 0 && \quad \text{on } \partial\Omega_{0}\times\left(0,T\right) \\
		\hl{\nabla\left(l_{0}^{4}\Delta c-2l_{0}^{2}c\right)\cdot\mathbf{N}} &= 0 && \quad \text{on } \partial\Omega_{0}\times\left(0,T\right) \\
\mathcal{H} &= \mathcal{H}_{0} && \quad \text{on } \Omega_{0}\times0  
	\end{alignedat}\right.
\end{equation}
Note that the governing equations for brittle fracture could also have been derived using the approach based on the balance laws, especially the microforce balance. For a more detailed derivation, it is referred to \citep{11_PF_DissBorden}. \hl{initial condition right?! or u0 v0 sufficient? -> dissertation Borden}
