\section{Numerical formulation} \label{sec:num_formul}
Formulas \eqref{eq:stress_equil} and \eqref{eq:c2_equil} or \eqref{eq:c4_equil}, respectively, represent the strong form of the given problem considering brittle fracture. As in \secref{sec:formul}, first of all brittle fracture is considered and the weak form is derived. Subsequently, the same is shown for ductile fracture. At the end, an Isogeometric Analysis framework as well as a time integration scheme is presented.

\subsection{Weak and semidiscrete Galerkin form} \label{sec:weak_Gal_form}
For the derivation of the weak form, at first suitable function spaces have to be defined:
\begin{equation} \label{eq:fct_spaces}
	\begin{aligned}
		\bm{\mathcal{S}} &= \{\mathbf{u}\left(t\right)\in\left(H^{1}\left(\Omega\right)\right)^{d}|\mathbf{u}\left(t\right)=\mathbf{g} \text{ on }\partial\Omega_{\mathbf{g}}\}, \\
		\bm{\mathcal{V}} &= \{\mathbf{w}\in\left(H^{1}\left(\Omega\right)\right)^{d}|\mathbf{u}\left(t\right)=\mathbf{0} \text{ on }\partial\Omega_{\mathbf{g}}\}, \\		
		\tilde{\mathcal{S}}_{n} &= \{c\left(t\right)\in H^{\frac{n}{2}}\left(\Omega\right)\}, \\
		\tilde{\mathcal{V}}_{n} &= \{q\in H^{\frac{n}{2}}\left(\Omega\right)\}.
	\end{aligned}
\end{equation}

\subsubsection{Brittle fracture} \label{sec:weak_brittle}
So as to get to the weak form, the governing equations have to be multiplied with testing functions, the divergence theorem has to be applied and boundary conditions need to be plugged in. Taking the testing functions from the spaces $\mathbf{\mathcal{V}}$ and $\tilde{\mathcal{V}}_{n}$, respectively, leads to the following weak forms for the second- and fourth-order phase-field theory for brittle fracture:

\textbf{n=2:}$\quad$ Find $\mathbf{u}\left(t\right)\in\bm{\mathcal{S}}$ and $c\left(t\right)\in\tilde{\mathcal{S}}_{2},t\in\left[0,T\right]$ so that $\forall \mathbf{w}\in\bm{\mathcal{V}}$ and $\forall q\in\tilde{\mathcal{V}}_{2}$:
\begin{equation} \label{eq:weak_2}
\begin{aligned}
\left\{\begin{alignedat}{2}
	\left(\rho \ddot{\mathbf{u}},\mathbf{w}\right)_{\Omega} + \left(\bm{\sigma},\nabla\mathbf{w}\right)_{\Omega} &= \left(\mathbf{h},\mathbf{w}\right)_{\partial\Omega_{\mathbf{h}}} + \left(\mathbf{b},\mathbf{w}\right)_{\partial\Omega_{\mathbf{g}}} \\
	\left(\left(\frac{4l_{0}\mathcal{H}}{\mathcal{G}_{c}}+1\right)c,q\right)_{\Omega} + \left(4l_{0}^{2}\nabla c,\nabla q\right)_{\Omega} &= \left(1,q\right)_{\Omega} \\
	\left(\rho\mathbf{u}\left(0\right),\mathbf{w}\right)_{\Omega} &= \left(\rho\mathbf{u}_{0},\mathbf{w}\right)_{\Omega} \\
	\left(\rho\dot{\mathbf{u}}\left(0\right),\mathbf{w}\right)_{\Omega} &= \left(\rho\dot{\mathbf{u}}_{0},\mathbf{w}\right)_{\Omega} \\
	\left(c\left(0\right),q\right)_{\Omega} &= \left(c_{0},q\right)_{\Omega}
\end{alignedat}\right.
\end{aligned}
\end{equation}

\textbf{n=4:}$\quad$ Find $\mathbf{u}\left(t\right)\in\bm{\mathcal{S}}$ and $c\left(t\right)\in\tilde{\mathcal{S}}_{4},t\in\left[0,T\right]$ so that $\forall \mathbf{w}\in\bm{\mathcal{V}}$ and $\forall q\in\tilde{\mathcal{V}}_{4}$:
\begin{equation} \label{eq:weak_4}
\begin{aligned}
\left\{\begin{alignedat}{2}
	\left(\rho \ddot{\mathbf{u}},\mathbf{w}\right)_{\Omega} + \left(\bm{\sigma},\nabla\mathbf{w}\right)_{\Omega} &= \left(\mathbf{h},\mathbf{w}\right)_{\partial\Omega_{\mathbf{h}}} + \left(\mathbf{b},\mathbf{w}\right)_{\partial\Omega_{\mathbf{g}}} \\
	\left(\left(\frac{4l_{0}\mathcal{H}}{\mathcal{G}_{c}}+1\right)c,q\right)_{\Omega} + \left(2l_{0}^{2}\nabla c,\nabla q\right)_{\Omega} + \left(l_{0}^{4}\Delta c,\Delta q\right)_{\Omega}&= \left(1,q\right)_{\Omega} \\
	\left(\rho\mathbf{u}\left(0\right),\mathbf{w}\right)_{\Omega} &= \left(\rho\mathbf{u}_{0},\mathbf{w}\right)_{\Omega} \\
	\left(\rho\dot{\mathbf{u}}\left(0\right),\mathbf{w}\right)_{\Omega} &= \left(\rho\dot{\mathbf{u}}_{0},\mathbf{w}\right)_{\Omega} \\
	\left(c\left(0\right),q\right)_{\Omega} &= \left(c_{0},q\right)_{\Omega}
\end{alignedat}\right.
\end{aligned}
\end{equation}
with $\left(\cdot,\cdot\right)_{\Omega}$ denoting the inner product with respect to the $L^{2}$-norm.

So as to solve problems \eqref{eq:weak_2} and \eqref{eq:weak_4}, it is necessary to choose finite-dimensional subspaces to the function spaces mentioned in \eqref{eq:fct_spaces}. Let $\bm{\mathcal{S}}^{h}\subset\bm{\mathcal{S}}$, $\bm{\mathcal{V}}^{h}\subset\bm{\mathcal{V}}$, $\tilde{\mathcal{S}}_{n}^{h}\subset\tilde{\mathcal{S}}_{n}$, $\tilde{\mathcal{V}}_{n}^{h}\subset\tilde{\mathcal{V}}_{n}$ and let $N_{A}\left(\mathbf{x}\right)$ denote the basis functions within the Finite Element approximation. Furthermore, the required functions within an element (with $n_{\text{nodes}}$ number of nodes) are approximated as linear combinations of these basis functions:
\begin{equation} \label{eq:fct_approx}
	\begin{aligned}
	\begin{alignedat}{2}
		\mathbf{u}\left(\mathbf{x},t\right) &\approx \mathbf{u}^{h}\left(\mathbf{x},t\right) &&= \sum\limits_{A=1}^{n_{\text{nodes}}}N_{A}\left(\mathbf{x}\right)\mathbf{u}_{A}\left(t\right), \\
		\mathbf{w}\left(\mathbf{x},t\right) &\approx \mathbf{w}^{h}\left(\mathbf{x},t\right) &&= \sum\limits_{A=1}^{n_{\text{nodes}}}N_{A}\left(\mathbf{x}\right)\mathbf{w}_{A}\left(t\right), \\
		c\left(\mathbf{x},t\right) &\approx c^{h}\left(\mathbf{x},t\right) &&= \sum\limits_{A=1}^{n_{\text{nodes}}}N_{A}\left(\mathbf{x}\right)c_{A}\left(t\right), \\
		q\left(\mathbf{x},t\right) &\approx q^{h}\left(\mathbf{x},t\right) &&= \sum\limits_{A=1}^{n_{\text{nodes}}}N_{A}\left(\mathbf{x}\right)q_{A}\left(t\right).
	\end{alignedat}
	\end{aligned}
\end{equation}
In order to obtain the semidiscrete Galerkin form of the problem, these approximations $\left(\cdot\right)\left(\mathbf{x},t\right)\approx\left(\cdot\right)^{h}\left(\mathbf{x},t\right)$ have to be plugged in into \eqref{eq:weak_2} and \eqref{eq:weak_4}. Changing the function spaces in these formulations and inserting \eqref{eq:fct_approx} will lead to the semidiscrete Galerkin form. These forms are not shown explicitly here due to the fact that only the superscript $\left(\cdot\right)^{h}$ has to be written next to the functions and function spaces. Apart from that, they are identical to the weak forms.

$\Gamma$-convergence has already been shown for the second-order phase-field theory but it has not been proved for the fourth-order theory yet. However, \citet{02_PF_HO_brittle} showed that the latter is well behaved in numerical examples. Comments on existing results for the higher-order model will shortly be outlined in \secref{sec:concl}.

\subsubsection{Ductile fracture} \label{sec:weak_ductile}
Analogously, the weak form for the second-order phase-field theory for ductile fracture can be derived. Considering the critical fracture energy density $\mathcal{G}_{c}=\frac{\mathcal{G}_{c}^{0}}{J}$ in the current configuration with Jacobian $J$, the system reads:

\textbf{n=2:}$\quad$ Find $\mathbf{u}\left(t\right)\in\bm{\mathcal{S}}$ and $c\left(t\right)\in\tilde{\mathcal{S}}_{2},t\in\left[0,T\right]$ so that $\forall \mathbf{w}\in\bm{\mathcal{V}}$ and $\forall q\in\tilde{\mathcal{V}}_{2}$:
\begin{equation} \label{eq:weak_2_ductile}
\begin{aligned}
\left\{\begin{alignedat}{1}
	\left(\rho \ddot{\mathbf{u}},\mathbf{w}\right)_{\Omega} + \left(\bm{\sigma},\nabla\mathbf{w}\right)_{\Omega} &= \left(\mathbf{h},\mathbf{w}\right)_{\partial\Omega_{\mathbf{h}}} + \left(\mathbf{b},\mathbf{w}\right)_{\partial\Omega_{\mathbf{g}}} \\
	\left(\dfrac{2l_{0}}{J\mathcal{G}_{c}}\left(\beta_{e}g'\left(c\right)+\beta_{p}g_{p}'\left(c\right)\left<\Psi_{p}-\Psi_{0}\right>\right)+c,q\right)_{\Omega} & \\
	+\left(4l_{0}^{2} c_{,\mathbf{X}}, q_{,\mathbf{X}}\right)_{\Omega} &= \left(1,q\right)_{\Omega} \\
	\left(\rho\mathbf{u}\left(0\right),\mathbf{w}\right)_{\Omega} &= \left(\rho\mathbf{u}_{0},\mathbf{w}\right)_{\Omega} \\
	\left(\rho\dot{\mathbf{u}}\left(0\right),\mathbf{w}\right)_{\Omega} &= \left(\rho\dot{\mathbf{u}}_{0},\mathbf{w}\right)_{\Omega} \\
	\left(c\left(0\right),q\right)_{\Omega_{0}} &= \left(c_{0},q\right)_{\Omega_{0}}.
\end{alignedat}\right.
\end{aligned}
\end{equation}
The semidiskrete Galerkin form follows as already mentioned for the weak form considering brittle fracture (see \secref{sec:weak_brittle}). Note that derivatives with respect to the reference configuration can be mapped to derivatives within the current configuration by applying the chain rule $\left(\cdot\right)_{,\mathbf{X}}=\left(\cdot\right)_{,\mathbf{x}}\mathbf{F}$. Thus, in the weak form the term $\left(4l_{0}^{2} c_{,\mathbf{X}}, q_{,\mathbf{X}}\right)_{\Omega}$ from \eqref{eq:weak_2_ductile} becomes $\left(4l_{0}^{2} \mathbf{F}^{T}c_{,\mathbf{x}}^{h}, \mathbf{F}^{T}q_{,\mathbf{x}}^{h}\right)_{\Omega}$ in the semidiscrete Galerkin form.

For the fourth-order phase-field theory, the weak form considering ductile fracture reads:

\textbf{n=4:}$\quad$ Find $\mathbf{u}\left(t\right)\in\bm{\mathcal{S}}$ and $c\left(t\right)\in\tilde{\mathcal{S}}_{2},t\in\left[0,T\right]$ so that $\forall \mathbf{w}\in\bm{\mathcal{V}}$ and $\forall q\in\tilde{\mathcal{V}}_{2}$:
\begin{equation} \label{eq:weak_4_ductile}
\begin{aligned}
\left\{\begin{alignedat}{1}
	\left(\rho \ddot{\mathbf{u}},\mathbf{w}\right)_{\Omega} + \left(\bm{\sigma},\nabla\mathbf{w}\right)_{\Omega} &= \left(\mathbf{h},\mathbf{w}\right)_{\partial\Omega_{\mathbf{h}}} + \left(\mathbf{b},\mathbf{w}\right)_{\partial\Omega_{\mathbf{g}}} \\
	\left(\dfrac{2l_{0}}{J\mathcal{G}_{c}}\left(\beta_{e}g'\left(c\right)+\beta_{p}g_{p}'\left(c\right)\left<\Psi_{p}-\Psi_{0}\right>\right)+c,q\right)_{\Omega} & \\
	+\left(2l_{0}^{2} c_{,\mathbf{X}}, q_{,\mathbf{X}}\right)_{\Omega} + \left(l_{0}^{4}\Delta^{\mathbf{X}} c,\Delta^{\mathbf{X}} q\right)_{\Omega} &= \left(1,q\right)_{\Omega} \\
	\left(\rho\mathbf{u}\left(0\right),\mathbf{w}\right)_{\Omega} &= \left(\rho\mathbf{u}_{0},\mathbf{w}\right)_{\Omega} \\
	\left(\rho\dot{\mathbf{u}}\left(0\right),\mathbf{w}\right)_{\Omega} &= \left(\rho\dot{\mathbf{u}}_{0},\mathbf{w}\right)_{\Omega} \\
	\left(c\left(0\right),q\right)_{\Omega_{0}} &= \left(c_{0},q\right)_{\Omega_{0}}
\end{alignedat}\right.
\end{aligned}
\end{equation} 
where $\Delta^{\mathbf{X}}$ refers to the Laplacian with respect to the reference configuration. Note that $\Delta^{\mathbf{X}}c=\sum\limits_{i}c_{,ii}F_{ii}^{2}$.

\subsection{Spatial discretization} \label{sec:space_discr}
Regarding the function spaces defined in \eqref{eq:fct_spaces}, especially the ones for the phase-field and the corresponding variations $\tilde{\mathcal{S}}_{n}$ and $\tilde{\mathcal{V}}_{n}$ (or the corresponding subspaces $\tilde{\mathcal{S}}_{n}^{h}$ and $\tilde{\mathcal{V}}_{n}^{h}$, respectively), there is a difficulty occurring within the Finite Element context. The definition of a Sobolev-Space reads
\begin{equation} \label{eq:Hilbertspace}
	H^{m}\left(\Omega\right) = \{u\in L^{2}\left(\Omega\right)|D^{\alpha}u\in L^{2}\left(\Omega\right),|\alpha|\leq m\}
\end{equation}
with $m\geq0$. In general, the solutions and its derivatives of Finite Element Analysis are continuous within an element. Unfortunately, they are generally not continuous between elements and thus, global continuity is lost. But for the fourth-order phase-field theory with $n=4$, a higher regularity is required. As a consequence, the traditional $C^{0}$ Finite Elements cannot be used. Using Isogeometric Finite Elements can circumvent this problem. Isogeometric Analysis (IGA) allows for an exact representation of geometries, even spherical or cylindrical shapes can be represented exactly which is not possible using standard Finite Elements. In general, the isogeometric basis is smooth which overcomes the problem of the necessity of a higher regularity and which leads to a more accurate stress representation \cite{01_PF_dyn_brittle}.

The main idea is to use the same basis for both, the analysis and the description of the model's geometric domain. Herefore, non-uniform rational B-Splines (NURBS) or T-Splines are used so as to compute with a smooth basis. Standard Finite Elements may be continuous but generally, they are not smooth. Since CAD geometries are based on NURBS, mesh changes like local refinements preserve the exact CAD geometry and no mapping between CAD geometry and FE geometry has to be established. \citep{11_PF_DissBorden}

\hl{\text{ToD0: Necessity for IGA and a \textbf{short} description.}}


\subsection{Time discretization} \label{sec:time_discr}
Since this work deals with dynamic fracture, there is a need for a time integration scheme. Here, a monolithic gernalized-$\alpha$ method is presented. This has been proposed by \citet{10_PF_genAlpha} in 1993. Advantages of this method are its second order accuracy as well as unconditonal stability provided that the parameters are well chosen. It is an implicit method which is outlined in this section. The notation used here is taken from the work of \citet{01_PF_dyn_brittle}.

On the basis of $\eqref{eq:weak_2}_{1}$, $\eqref{eq:weak_4}_{1}$, $\eqref{eq:weak_2_ductile}_{1}$ or $\eqref{eq:weak_4_ductile}_{1}$, respectively, and \eqref{eq:fct_approx}, let the residual vectors for the displacement be defined as
\begin{equation} \label{eq:res_vecs_u}
	\begin{aligned}
		\mathbf{R}^{u}&=\{R_{A,i}^{u}\}, \\
		R_{A,i}^{u} &= \left(\mathbf{h},N_{A}\mathbf{e}_{i}\right)_{\partial\Omega_{\mathbf{h}}} + \left(\mathbf{b},N_{A}\mathbf{e}_{i}\right)_{\partial\Omega_{\mathbf{g}}} - \left(\rho\ddot{\mathbf{u}}^{h},N_{A}\mathbf{e}_{i}\right)_{\Omega}-\left(\sigma_{jk},B_{A}^{ijk}\right)_{\Omega},
	\end{aligned}
\end{equation}
with $i$'th Euclidean basis vector $\mathbf{e}_{i}$ and
\begin{equation}
	B_{A}^{ijk}=\frac{1}{2}\left(\frac{\partial N_{A}}{\partial x_{j}}\delta_{ik}+\frac{\partial N_{A}}{\partial x_{k}}\delta_{ij}\right)
\end{equation}
so that $\epsilon_{ijk}=\sum_{A}B_{A}^{ijk}d_{A,i}$.

For the second-order phase-field theory for brittle fracture, the residual vectors based on $\eqref{eq:weak_2}_{2}$ are given by
\begin{equation} \label{eq:res_vecs_c2}
	\begin{aligned}
		\mathbf{R}^{c}&=\{R_{A}^{c}\}, \\
		R_{A}^{c} &= \left(1,N_{A}\right)_{\Omega} - \left(\left(\frac{4l_{0}\mathcal{H}}{\mathcal{G}_{c}}+1\right)c^{h},N_{A}\right)_{\Omega} - \left(4l_{0}^{2}c^{h}_{,i},N_{A,i}\right)_{\Omega}.
	\end{aligned}
\end{equation}

For the fourth-order phase-field theory for brittle fracture, these residual vectors based on $\eqref{eq:weak_4}_{2}$ change to:
\begin{equation} \label{eq:res_vecs_c4}
	\begin{aligned}
		\mathbf{R}^{c}&=\{R_{A}^{c}\}, \\
		R_{A}^{c} &= \left(1,N_{A}\right)_{\Omega} - \left(\left(\frac{4l_{0}\mathcal{H}}{\mathcal{G}_{c}}+1\right)c^{h},N_{A}\right)_{\Omega} - \left(2l_{0}^{2}c^{h}_{,i},N_{A,i}\right)_{\Omega} - \left(l_{0}^{4}c^{h}_{,ii},N_{A,jj}\right)_{\Omega}.
	\end{aligned}
\end{equation}

Analogously, for the second-order phase-field theory for ductile fracture, the residual vectors based on $\eqref{eq:weak_2_ductile}_{2}$ are given by
\begin{equation} \label{eq:res_vecs_c2_ductile}
	\begin{aligned}
	\begin{alignedat}{2}
		\mathbf{R}^{c}&=\{R_{A}^{c}\}, \\
		R_{A}^{c} &= \left(1,N_{A}\right)_{\Omega} &&- \left(\dfrac{2l_{0}}{J\mathcal{G}_{c}}\left(\beta_{e}g'\left(c^{h}\right)+\beta_{p}g_{p}'\left(c^{h}\right)\left<\Psi_{p}-\Psi_{0}\right>\right)+c^{h},N_{A}\right)_{\Omega} \\
			& &&- \left(4l_{0}^{2}c^{h}_{,I},N_{A,I}\right)_{\Omega}
			\end{alignedat}
	\end{aligned}
\end{equation}
and for the corresponding fourth-order theory based on $\eqref{eq:weak_4_ductile}_{2}$ by
\begin{equation} \label{eq:res_vecs_c4_ductile}
	\begin{aligned}
	\begin{alignedat}{2}
		\mathbf{R}^{c}&=\{R_{A}^{c}\}, \\
		R_{A}^{c} &= \left(1,N_{A}\right)_{\Omega} &&- \left(\dfrac{2l_{0}}{J\mathcal{G}_{c}}\left(\beta_{e}g'\left(c^{h}\right)+\beta_{p}g_{p}'\left(c^{h}\right)\left<\Psi_{p}-\Psi_{0}\right>\right)+c^{h},N_{A}\right)_{\Omega} \\
		 & &&- \left(2l_{0}^{2}c^{h}_{,I},N_{A,I}\right)_{\Omega} - \left(l_{0}^{4}c^{h}_{,II},N_{A,JJ}\right)_{\Omega}.
	\end{alignedat}
	\end{aligned}
\end{equation}

Considering time step $n$, time step size $\Delta t = t_{n+1}-t_{n}$, parameters $\alpha_{f},\alpha_{m},\beta,\gamma$ as well as vectors $\mathbf{u}_{n}$ and $\mathbf{c}_{n}$ for the control variable degrees-of-freedom of the displacements and phase-field (see \eqref{eq:fct_approx}), $\textbf{v}_{n}=\dot{\mathbf{u}}_{n}$ and $\mathbf{a}_{n}=\ddot{\mathbf{u}}_{n}$ are defined. The method then reads as follows:

Given $\mathbf{u}_{n}$, $\mathbf{v}_{n}$ and $\mathbf{a}_{n}$. Find $\mathbf{u}_{n+1}$, $\mathbf{v}_{n+1}$, $\mathbf{a}_{n+1}$, $\mathbf{u}_{n+\alpha_{f}}$, $\mathbf{v}_{n+\alpha_{f}}$, $\mathbf{a}_{n+\alpha_{m}}$ and $\mathbf{c}_{n+1}$ so that:
\begin{equation} \label{eq:gen_alpha}
\begin{aligned}
\begin{alignedat}{1}
        \mathbf{R}^{u}\left(\mathbf{u}_{n+\alpha_{f}},\mathbf{v}_{n+\alpha_{f}},\mathbf{a}_{n+\alpha_{m}},\mathbf{c}_{n+1}\right) &= 0, \\
    	\mathbf{R}^{c}\left(\mathbf{u}_{n+\alpha_{f}},\mathbf{c}_{n+1}\right) &= 0, \\
        \mathbf{u}_{n+\alpha_{f}} &= \mathbf{u}_{n}+\alpha_{f}\left(\mathbf{u}_{n+1}-\mathbf{u}_{n}\right), \\      
        \mathbf{v}_{n+\alpha_{f}} &= \mathbf{v}_{n}+\alpha_{f}\left(\mathbf{v}_{n+1}-\mathbf{v}_{n}\right), \\
        \mathbf{a}_{n+\alpha_{m}} &= \mathbf{a}_{n}+\alpha_{m}\left(\mathbf{a}_{n+1}-\mathbf{a}_{n}\right), \\
        \mathbf{v}_{n+1} &= \mathbf{v}_{n}+\Delta t\left(\left(1-\gamma\right)\mathbf{a}_{n}+\gamma\mathbf{a}_{n+1}\right), \\  
        \mathbf{u}_{n+\alpha f} &= \mathbf{u}_{n}+\alpha_{f}\left(\mathbf{u}_{n+1}-\mathbf{u}_{n}\right).
\end{alignedat}
\end{aligned}
\end{equation}
These equations directly show that this method is an implicit one. Thus, at each time step a Newton-Raphson iteration is required so as to solve the nonlinear equations \eqref{eq:gen_alpha}:

Until convergence is reached, solve the linearized system:
\begin{equation} \label{eq:NewtonRaphson}
	\begin{aligned}
	\begin{alignedat}{2}
		\dfrac{\partial\mathbf{R}_{i}^{u}}{\partial\mathbf{a}_{n+1}}\Delta \mathbf{a} &+ \dfrac{\partial\mathbf{R}_{i}^{u}}{\partial\mathbf{c}_{n+1}}\Delta\mathbf{c} &&= -\mathbf{R}_{i}^{u}, \\
		\dfrac{\partial\mathbf{R}_{i}^{c}}{\partial\mathbf{a}_{n+1}}\Delta \mathbf{a} &+ \dfrac{\partial\mathbf{R}_{i}^{c}}{\partial\mathbf{c}_{n+1}}\Delta\mathbf{c} &&= -\mathbf{R}_{i}^{c},
	\end{alignedat}
	\end{aligned}
\end{equation}
with $i$ denoting the current Newton-Raphson iteration and $\mathbf{R}_{i}^{u}$ and $\mathbf{R}_{i}^{c}$ being the current residuals (see $\eqref{eq:gen_alpha}_{1-2}$). Following \citet{01_PF_dyn_brittle}, convergence can be for instance defined by
\begin{equation} \label{eq:conv_cond}
	\max\{\frac{||\mathbf{R}_{i}^{u}||_{L^{2}}}{||\mathbf{R}_{0}^{u}||_{L^{2}}},\frac{||\mathbf{R}_{i}^{c}||_{L^{2}}}{||\mathbf{R}_{0}^{c}||_{L^{2}}}\}\leq tol
\end{equation}
with $tol$ for example set to $10^{-4}$.

According to \citet{10_PF_genAlpha} the parameters should be chosen as follows:
\begin{equation}
	\begin{aligned}
		\alpha_{f} &= \frac{1}{\rho_{\infty}+1}, \\
		\alpha_{m} &= \frac{2-\rho_{\infty}}{\rho_{\infty}+1}, \\
		\beta &= \frac{1}{4}\left(1+\alpha_{m}-\alpha_{f}\right)^{2}, \\
		\gamma &= \frac{1}{2}+\alpha_{m}-\alpha_{f},
	\end{aligned}
\end{equation}
with spectral radius $\rho_{\infty}$ of the amplification matrix at $\Delta t=\infty$. Thus, second-order accuracy and unconditional stability are reached for a second-order linear problem. 

\hl{\text{What is the meaning of } \rho_{\infty} \text{?}}
