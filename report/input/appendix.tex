\appendix

\section{Sharp interface models} \label{appsec:sharp}

\subsection{Virtual crack closure technique} \label{appsec:virtClos}
Assuming a two-dimensional plane stress and plane strain model, the virtual crack closure technique represents a crack by introducing one-dimensional discontinuities by a line of nodes. The total energy release rate is computed locally at the crack front. If two Finite Elements shall be split by a crack, these elements will not share the same edge and nodes anymore. Thus, new nodes or edges have to be created. So as to handle this problem, there are nodes with identical coordinates at regions where cracks form. If there is no crack yet, these nodes will lie over each other and a multipoint constraint is used so as to keep them at the same position. As soon as the respective Finite Elements shall be split, this constraint is disregarded and the two nodes can move apart from each other. Consequently, the elements move apart and a crack within the mesh is introduced. In order to avoid kinematically incompatible self-penetration of elements, an element-wise opening instead of a node-wise opening is required. As shown by \citet{03_SotA_virtClos}, the calculation of the strain energy release rates, which determine crack nucleation, depend on the element type. Therefore, the formulas used in the computation have to be adapted for different elements shapes and varying number of nodes per element. Since these formulas quickly get very complicated, especially for three-dimensional problems, the implementation requires high efforts. Additionally, correction formulas for sharp corners and different element length are required. \cite{03_SotA_virtClos} 

\subsection{Cohesive segments method} \label{appsec:cohes}
In the cohesive segments method the crack is represented by a set of overlapping cohesive segments. Thus, the separation process is limited to a set of discrete planes in 3D or discrete lines in 2D, respectively. As soon as a stress state violates a given fracture criterion, a new segment is created. Such a segment adds a discontinuity to an element during calculations by exploiting the partition-of-unity property. For all nodes whose support is cut by a crack into two disjoint pieces or which are crossed by a cohesive segment, respectively, an additional equilibrium equation is given and new degrees of freedom are added. The latter do not affect energy conservation and describe the magnitude of the displacement jump. Within the FE-approximation, this engenders that the approximate displacement field $\mathbf{u}^{h}=\sum_{i}\mathbf{u}_{i}N_{i}$, where $N_{i}$ denote the used shape functions, is enriched by a discontinuity function and asymptotic crack tip functions which are necessary if a crack does not begin on an edge. However, all this leads to several disadvantages: Quadrature rules have to be adapted, nodes need to have a variable number of degrees of freedom, there is a high accuracy error, the stress state at the tip of a segment is not well described and there is a problem is the specification of the geometry of the cohesive surfaces. \cite{02_SotA_cohes}\cite{01_SotA_cohes_dyn}

\section{Ductile fracture} \label{appsec:ductile}