\appendix

\section{Sharp interface models} \label{appsec:sharp}
In this section, shortly, two examples for sharp interface models are illustrated.

\subsection{Virtual crack closure technique} \label{appsec:virtClos}
Assuming a two-dimensional plane stress or plane strain model, the \textit{virtual crack closure technique} represents a crack by introducing one-dimensional discontinuities by a line of nodes. The total energy release rate is computed locally at the crack front. If two Finite Elements shall be split by a crack, these elements will not share the same edge and nodes anymore. Thus, new nodes or edges have to be created. So as to handle this problem, there are nodes with identical coordinates at regions where cracks form. If there is no crack yet, these nodes will lie over each other and a multipoint constraint is imposed so as to keep them at the same position. As soon as the respective Finite Elements shall be split, this constraint is disregarded and the two nodes can move apart from each other. Consequently, the elements move apart and a crack within the mesh is introduced. In order to avoid kinematically incompatible self-penetration of elements, an element-wise opening instead of a node-wise opening is required. As shown by \citet{03_SotA_virtClos}, the calculation of the strain energy release rates, which determine crack nucleation, depend on the element type. Therefore, the formulas used in the computation have to be adapted for different element shapes and varying number of nodes per element. Since these formulas quickly get very complicated, especially for three-dimensional problems, the implementation requires high efforts. Additionally, correction formulas for sharp corners and different element length are required. \cite{03_SotA_virtClos} 

\subsection{Cohesive segments method} \label{appsec:cohes}
In the \textit{cohesive segments method}, the crack is represented by a set of overlapping cohesive segments. Thus, the separation process is limited to a set of discrete planes in 3D or discrete lines in 2D, respectively. As soon as a stress state violates a given fracture criterion, a new segment is created. Such a segment adds a discontinuity to an element during calculations by exploiting the partition-of-unity property. For all nodes whose support is cut by a crack into two disjoint pieces or which are crossed by a cohesive segment, respectively, an additional equilibrium equation is given and new degrees of freedom are added. The latter do not affect energy conservation and describe the magnitude of the displacement jump. Within the FE-approximation, this engenders that the approximate displacement field $\mathbf{u}^{h}=\sum_{A}N_{A}\mathbf{u}_{A}$, where $N_{A}$ denote the shape functions, is enriched by a discontinuity function and asymptotic crack tip functions which are necessary if a crack does not begin on an edge. However, all this leads to several disadvantages: Quadrature rules have to be adapted, nodes need to have a variable number of degrees of freedom, there is a high accuracy error, the stress state at the tip of a segment is not well described and the specification of the geometry of the cohesive surfaces might lead to problems. \cite{02_SotA_cohes}\cite{01_SotA_cohes_dyn}

\section{Ductile fracture} \label{appsec:ductile}
Already in \secref{sec:ductile_frac}, the differences between modelling brittle and ductile fracture have been outlined. In the following, energy laws are derived and put together to a thermodynamically consistent formulation for ductile fracture.

\citet{06_PF_ductile} proposes a method similar to the one presented for brittle fracture. The major change lies in the Helmholtz free energy (see \eqref{eq:Helmholtz}) which is replaced by
\begin{equation} \label{eq:Helmholtz_ductile1}
	\Psi_{n} = g\left(c,p\right)\Psi_{e}^{+}\left(\bm{\varepsilon}\right)+\Psi_{e}^{-}\left(\bm{\varepsilon}\right)+\Psi_{p}\left(\alpha\right)+\mathcal{G}_{c}\Gamma_{c,n}
\end{equation}
with the new degradation function $g\left(c,p\right)=c^{2p}+\eta$ and $p=\frac{\epsilon_{eq}^{p}}{\epsilon_{eq,crit}^{p}}$, $\epsilon_{eq}^{p}\left(t\right)=\sqrt{\frac{2}{3}}\int\limits_{0}^{t}\sqrt{\dot{\bm{\varepsilon}}:\dot{\bm{\varepsilon}}}\mathrm{d}t$. The plastic energy density function considering linear isotropic hardening reads $\Psi_{p}\left(\alpha\right)=\sigma_{y}\alpha+\frac{1}{2}h\alpha^{2}$ with yield stress $\sigma_{y}$, hardening variable $\alpha$ and hardening modulus $h>0$. The idea behind this new degradation function is that $c$ depends on $p$ so that the fracture process can be seen as the accumulation of ductile damage. $\Psi_{p}$ will take the dominating role over $\Psi_{e}^{+}$ and thus, a plasticity-driven fracture initiation is achieved. However, here a more abstract derivation of the governing equations will be presented which is based on the work of \citet{03_PF_ductile}.

In \secref{sec:formul_Griffith} small deformations and deformation gradients have been assumed. Since this does not hold true for ductile fracture (see \figref{fig:plastic}), another approach for the derivation of the strong form is chosen. As \citet{03_PF_ductile} have outlined, previous approaches on ductile fracture suffer in the fact that yield surface and hardening modulus are not effected by the phase-field's evolution. Thus, at some point plastic strains saturate and deformation is again dominated by recoverable elastic strains. The notation mentioned at the beginning of this section stays the same. The deformation gradient, which cannot be considered small anymore, now reads
\begin{equation} \label{eq:deform_grad}
	\mathbf{F}=\dfrac{\partial\varphi\left(\mathbf{X},t\right)}{\partial\mathbf{X}}
\end{equation}
with deformation map $\varphi:\left(\Omega_{0}\times\mathbb{R}\right)\rightarrow\mathbb{R}^{d}$, body $\Omega_{0}$ in the reference configuration  and $\mathbf{x}=\varphi\left(\mathbf{X},t\right)$. \eqref{eq:deform_grad} is now decomposed into the elastic and plastic deformations gradient:
\begin{equation} \label{eq:deform_grad_decomp}
	\mathbf{F}=\mathbf{F}^{e}\mathbf{F}^{p}.
\end{equation}
In contrast to \eqref{eq:Helmholtz}, the new stored energy functional proposed by \citet{03_PF_ductile} reads
\begin{equation} \label{eq:fctal_ductile}
	\tilde{\Psi}\left(\mathbf{F},c,c_{,\mathbf{X}},\mathbf{F}^{p},\alpha\right) = \int\limits_{\Omega_{0}}\left[g\left(c\right)\Psi^{+}\left(\mathbf{F},\mathbf{F}^{p}\right)+\Psi^{-}\left(\mathbf{F},\mathbf{F}^{p}\right)+g_{p}\left(c\right)\Psi_{p}\left(\alpha\right)+\Gamma_{c,n}\right]\mathrm{d}\Omega_{0}
\end{equation}
with $\left(\cdot\right)_{\mathbf{X}}$ denoting the derivative with respect to $\mathbf{X}$.\footnote{Note that for ductile fracture in this work, $\nabla\left(\cdot\right)$ is replaced by the index notation $\left(\cdot\right)_{,\mathbf{X}}$ or $\left(\cdot\right)_{,\mathbf{x}}$, respectively, in order to distinguish between derivatives with respect to the reference or current configuration. For brittle fracture, everything has been derived within the current configuration.} As in the work by \citet{06_PF_ductile}, an effective plastic work contribution $\Psi_{p}$ depending on the internal hardening variable $\alpha$ has been added. The plastic degradation function $g_{p}\left(c\right)$ provides, analogous to $g\left(c\right)$, a mechanism for crack growth driven by the development of plastic strains.

\subsection{Balance laws and strong form} \label{appsec:balance_laws}
The approach for the derivation of the governing equations is based on balance laws. To keep things short, not every single step of their derivation but more likely their results are presented here. Microforce balance laws are assumed to model the phase-field's evolution within a large deformation setting. The balance of linear momentum and angular momentum in the reference configuration lead to (also see $\eqref{eq:stress_equil}_{1}$)
\begin{equation} \label{eq:lin_mom_ang_mom}
	\begin{aligned}
		Div\left(\mathbf{P}\right)+\mathbf{B}&=\rho_{0}\ddot{\mathbf{U}}, \\
		\bm{\sigma} &= \bm{\sigma}^{T}
	\end{aligned}
\end{equation}
with $1^{st}$ Piola-Kirchhoff stress tensor $\mathbf{P}$, density $\rho_{0}$ in the reference configuration and $Div\left(\cdot\right)$ denoting the divergence in the reference configuration. Analogous to this balance law, assuming an internal microforce $\pi\left(\mathbf{x},t\right)\in\mathbb{R}$, an external microforce $l\left(\mathbf{x},t\right)\in\mathbb{R}$ acting on the body and an external microforce $\lambda\left(\mathbf{x},t\right)=\bm{\xi}\cdot\mathbf{N}$ acting on the surface (with microforce traction vector $\bm{\xi}\left(\mathbf{x},t\right)\in\mathbb{R}^{d}$), the microforce balance leads to
\begin{equation} \label{eq:microforce_balance}
	Div\left(\bm{\xi}\right)+\pi+l=0.
\end{equation}
The energy balance with $e_{0}$ describing the internal energy per unit volume reads
\begin{equation} \label{eq:energy_balance}
	\dfrac{\mathrm{d}}{\mathrm{d}t}\int\limits_{\Omega_{0}}\left(\dfrac{1}{2}\rho_{0}|\dot{\mathbf{U}}|^{2}+e_{0}\right)\mathrm{d}\Omega_{0} = 
	\int\limits_{\partial\Omega_{0}}\left(\mathbf{P}\mathbf{N}\right)\cdot\dot{\mathbf{U}}\mathrm{d}\partial\Omega_{0}+\int\limits_{\Omega_{0}}\mathbf{B}\cdot\dot{\mathbf{U}}\mathrm{d}\Omega_{0}+\int\limits_{\partial\Omega_{0}}\lambda\dot{c}\text{ }\mathrm{d}\partial\Omega_{0}+\int\limits_{\Omega_{0}}l\dot{c}\text{ }\mathrm{d}\Omega_{0},
\end{equation}
where $\dot{c}$ is found to be the work conjugate of the microforces. The second law of thermodynamics leads to
\begin{equation} \label{eq:entropy_balance_tmp}
	\Theta\dot{s}\geq0
\end{equation}
with entropy per unit volume $s$ and absolute temperature $\Theta$. Note that heat fluxes and heat sources are not considered here (isothermal and adiabatic process). Using the dissipation inequality $e_{0}=\Psi+\Theta  s$ (see Legendre transformation), \eqref{eq:entropy_balance_tmp} can be rewritten as
\begin{equation} \label{eq:entropy_balance}
	\dot{\Psi}\leq\dot{e}_{0}
\end{equation}
with Helmholtz free energy $\Psi$. Regarding the microforces within the energy balance \eqref{eq:energy_balance}, the divergence theorem has to be applied by using that $\lambda\left(\mathbf{x},t\right)=\bm{\xi}\cdot\mathbf{N}$ and \eqref{eq:microforce_balance} has to be plugged in. For the remaining terms, integration by parts leads to 
\begin{equation} \label{eq:en_bal_tmp}
	\int\limits_{\partial\Omega_{0}}\left(\mathbf{P}\mathbf{N}\right)\cdot\dot{\mathbf{U}}\mathrm{d}\partial\Omega_{0}=\int\limits_{\Omega_{0}}Div\left(\mathbf{P}\right)\cdot\dot{\mathbf{U}}\mathrm{d}\Omega_{0}+\int\limits_{\Omega_{0}}\mathbf{P}:\dot{\mathbf{U}}_{,\mathbf{X}}\mathrm{d}\Omega_{0}
\end{equation}
with $\dot{\mathbf{U}}_{,\mathbf{X}}=\dot{\mathbf{F}}$. Inserting $\eqref{eq:lin_mom_ang_mom}_{1}$ into \eqref{eq:en_bal_tmp} and using that
\begin{equation}
	\dfrac{\mathrm{d}}{\mathrm{d}t}\left(\dfrac{1}{2}\rho_{0}|\dot{\mathbf{U}}|^{2}\right) = \rho_{0}\dot{\mathbf{U}}\cdot\ddot{\mathbf{U}}
\end{equation}
leads to the following important statement:

A thermodynamically consistent model is achieved as long as $\Psi$ fulfills
\begin{equation} \label{eq:thermodyn_cons}
		\int\limits_{\Omega_{0}}\dot{\Psi}\mathrm{d}\Omega_{0} \leq \int\limits_{\Omega_{0}}\left(\mathbf{P}:\dot{\mathbf{F}}+\bm{\xi}\cdot\dot{c}_{,\mathbf{X}}-\pi\dot{c}\right)\mathrm{d}\Omega_{0} = \int\limits_{\Omega_{0}}\left(\dfrac{1}{2}\mathbf{S}:\dot{\mathbf{C}}+\bm{\xi}\cdot\dot{c}_{,\mathbf{X}}\pi\dot{c}\right)\mathrm{d}\Omega_{0}.
\end{equation}
$\mathbf{S}$ denotes the $2^{nd}$ Piola-Kirchhoff stress tensor and $\mathbf{C}=\mathbf{F}^{T}\mathbf{F}$ the right Cauchy-Green tensor.

Assuming that $\Psi=\Psi\left(\mathbf{C},\mathbf{C}^{p},\mathbf{Q},c,c_{,\mathbf{X}},\dot{c}\right)$ with plastic right Cauchy-Green tensor $\mathbf{C}^{p}={\mathbf{F}^{p}}^{T}\mathbf{F}^{p}$ and set of internal plastic variables $\mathbf{Q}$, applying the chain rule in \eqref{eq:thermodyn_cons} leads to
\begin{equation} \label{eq:thermodyn_cons2}
	\begin{aligned}
	\int\limits_{\Omega_{0}}&\left(\dfrac{\partial\Psi}{\partial\mathbf{C}}:\dot{\mathbf{C}}+\dfrac{\partial\Psi}{\partial\mathbf{C}^{p}}:\dot{\mathbf{C}}^{p}+\dfrac{\partial\Psi}{\partial\mathbf{Q}}\cdot\dot{\mathbf{Q}}+\dfrac{\partial\Psi}{\partial c}\dot{c}+\dfrac{\partial\Psi}{\partial c_{,\mathbf{X}}}\cdot\dot{c}_{,\mathbf{X}}+\dfrac{\partial\Psi}{\partial\dot{c}}\ddot{c}\right)\mathrm{d}\Omega_{0} \\
	&\leq \int\limits_{\Omega_{0}}\left(\dfrac{1}{2}\mathbf{S}:\dot{\mathbf{C}}+\bm{\xi}\cdot\dot{c}_{,\mathbf{X}}-\pi\dot{c}\right)\mathrm{d}\Omega_{0}.
	\end{aligned}
\end{equation}
The plastic dissipation $\mathbb{D}^{p}=-\dfrac{\partial\Psi}{\partial\mathbf{C}^{p}}:\dot{\mathbf{C}}^{p}-\dfrac{\partial\Psi}{\partial\mathbf{Q}}\cdot\dot{\mathbf{Q}}$ can be set to zero for purely elastic deformations. Comparing the terms on the left hand and right hand side of \eqref{eq:thermodyn_cons2}, the following relations can be derived:
\begin{equation} \label{eq:relations_cons2}
	\begin{aligned}
		\dfrac{\partial\Psi}{\partial\dot{c}} &= 0, \\
		\dfrac{\partial\Psi}{\partial c_{,\mathbf{X}}} &= \bm{\xi}, \\
		2\dfrac{\partial\Psi}{\partial\mathbf{C}} &= \mathbf{S}, \\
		\left(\pi+\dfrac{\partial\Psi}{\partial c}\right)\dot{c} &\leq 0, \\
		\mathbb{D}^{p} &\geq 0.
	\end{aligned}
\end{equation}
$\eqref{eq:relations_cons2}_{4}$ can be rewritten as
\begin{equation} \label{eq:relation_cons_beta}
	\pi = -\dfrac{\partial\Psi}{\partial c}-\beta\dot{c}
\end{equation}
with function $\beta=\beta\left(\mathbf{C},\mathbf{C}^{p},c,c_{,\mathbf{X}}\right)\geq0$ \footnote{$\beta$ can be seen as a dissipation constant.}. Inserting \eqref{eq:relations_cons2} and \eqref{eq:relation_cons_beta} into the balance laws $\eqref{eq:lin_mom_ang_mom}_{1}$ and \eqref{eq:microforce_balance} leads to following equations (using $\mathbf{P}=\mathbf{F}\mathbf{S}$):
\begin{equation} \label{eq:balance_laws}
	\begin{aligned}
		Div\left(2\mathbf{F}\dfrac{\partial\Psi}{\partial\mathbf{C}}\right)+\mathbf{B} &= \rho_{0}\ddot{\mathbf{U}}, \\
		Div\left(\dfrac{\partial\Psi}{\partial c_{,\mathbf{X}}}\right)+l-\dfrac{\partial\Psi}{\partial c} &= \beta\dot{c}.
	\end{aligned}
\end{equation}
Considering \eqref{eq:fctal_ductile}, the Helmholtz free energy for ductile fracture reads
\begin{equation} \label{eq:Helmholtz_ductile2}
	\Psi_{n}\left(\mathbf{C},\mathbf{C}^{p},\mathbf{Q},c,c_{,\mathbf{X}}\right) = g\left(c\right)\Psi^{+}\left(\mathbf{C},\mathbf{C}^{p}\right)+\Psi^{-}\left(\mathbf{C},\mathbf{C}^{p}\right)+g_{p}\left(c\right)\Psi_{p}\left(\mathbf{Q}\right)+\Gamma_{c,n}.
\end{equation}
Setting $l=0$ and $\beta=0$ leads to the governing equations. The stress equilibrium is given by
\begin{equation} \label{eq:stress_equil_ductile}
	 \left\{\begin{alignedat}{2}
		Div\left(2\mathbf{F}\left(g\left(c\right)\dfrac{\partial\Psi^{+}}{\partial\mathbf{C}}+\dfrac{\partial\Psi^{-}}{\partial\mathbf{C}}\right)\right)+\mathbf{B} &= \rho_{0}\ddot{\mathbf{U}} && \quad\text{on } \Omega_{0}\times\left(0,T\right) \\
		\mathbf{U} &= \mathbf{G} && \quad\text{on } \partial\Omega_{0,\mathbf{G}}\times\left(0,T\right) \\
		\mathbf{T} &= \mathbf{H} && \quad\text{on } \partial\Omega_{0,\mathbf{H}}\times\left(0,T\right) \\
		\mathbf{U} &= \mathbf{U}_{0} && \quad\text{on } \Omega_{0}\times0 \\
		\dot{\mathbf{U}} &= \mathbf{V}_{0} && \quad\text{on } \Omega_{0}\times0
  \end{alignedat}\right.
\end{equation}
with Piola-Kirchhoff traction vector $\mathbf{T}=\mathbf{P}\mathbf{N}$. Considering the second-order phase-field theory and inserting the two partial derivatives $\frac{\partial\Psi}{\partial c_{,\mathbf{X}}}$ and $\frac{\partial\Psi}{\partial c}$ required in $\eqref{eq:balance_laws}_{2}$ with the help of \eqref{eq:Helmholtz_ductile2}, leads to following equation:
\begin{equation} \label{eq:c2_ph_evolution_tmp}
\dfrac{2l_{0}}{\mathcal{G}_{c}^{0}}\left(g'\left(c\right)\Psi^{+}+g_{p}'\left(c\right)\Psi_{p}\right) + c - 4l_{0}^{2}\Delta c = 1 \quad\text{on } \Omega\times\left(0,T\right).
\end{equation}
However, the irreversibility conditions outlined in \eqref{eq:KuhnTucker} have not been considered yet. Now, the history field is given by
\begin{equation} \label{eq:history_field_ductile}
	\mathcal{H}\left(\mathbf{C},\mathbf{C}^{p}\right) = \max\limits_{\bar{t}\leq t} \Psi^{+}\left(\mathbf{C}\left(\bar{t}\right),\mathbf{C}^{p}\left(\bar{t}\right)\right).
\end{equation}
Having $\left<\cdot\right>$ defined as in \eqref{eq:spectr_decomp}, the term $\Psi_{p}$ in \eqref{eq:c2_ph_evolution_tmp} is replaced by $\left<\Psi_{p}-\Psi_{0}\right>$. Actually, $\Psi_{p}$ is already assumed to be monotonically increasing so that no additional constraints are necessary in order to enforce irreversibility crack growth caused by plastic deformation. However, \citet{03_PF_ductile} introduce the plastic work threshold $\Psi_{0}$ so as to have more control over the contribution of plastic deformation to crack growth. Thus, if $\Psi_{p}<\Psi_{0}$, there will be no contribution of plastic deformation to crack growth. This point can now be controlled easily. Furthermore, \citet{03_PF_ductile} introduce parameters $\beta_{e}\in\left[0,1\right]$ and $\beta_{p}\left[0,1\right]$ which can be used in order to weight the contributions from elastic strain energy and plastic work to crack growth. These modifications, \eqref{eq:c2_ph_evolution_tmp} and \eqref{eq:history_field_ductile} lead to the governing equations for the evolution of the phase-field considering ductile fracture and the second-order phase-field theory:
\begin{equation} \label{eq:c2_equil_ductile}
	\left\{\begin{alignedat}{2}
		\dfrac{2l_{0}}{\mathcal{G}_{c}^{0}}\left(\beta_{e}g'\left(c\right)\mathcal{H}+\beta_{p}g_{p}'\left(c\right)\left<\Psi_{p}-\Psi_{0}\right>\right) + c - 4l_{0}^{2}\Delta c\ &= 1 && \quad\text{on } \Omega_{0}\times\left(0,T\right) \\
		c_{,\mathbf{X}}\cdot\mathbf{N} &= 0 && \quad \text{on } \partial\Omega_{0}\times\left(0,T\right) \\
		c &= c_{0} && \quad \text{on } \Omega_{0}\times0.
	\end{alignedat}\right.
\end{equation}
The boundary conditions on the phase-field result from the homogeneous natural boundary conditions arising from the derivation of the weak form \citep{11_PF_DissBorden}.

There are several constitutive models so as to determine the energy density functions $\Psi^{+}$, $\Psi^{-}$ and $\Psi_{p}$. These have been well outlined by \citet{03_PF_ductile} and within this paper, it is only referred to this work. 

Since $\eqref{eq:crack_dens_fctals}_{2}$ includes second order spatial derivatives of $c$ for the fourth-order phase-field theory, the relations \eqref{eq:relations_cons2} have to be adapted. For that, the Helmholtz free energy is assumed to take the form $\Psi=\Psi\left(\mathbf{C},\mathbf{C}^{p},\mathbf{Q},c,c_{,\mathbf{X}},c_{,ii},\dot{c}\right)$ so that \eqref{eq:thermodyn_cons2} changes to\footnote{Note that Einstein's summation convention on repeated indices is used here.}:
\begin{equation} \label{eq:thermodyn_cons4}
	\begin{aligned}
	\int\limits_{\Omega_{0}}&\left(\dfrac{\partial\Psi}{\partial\mathbf{C}}:\dot{\mathbf{C}}+\dfrac{\partial\Psi}{\partial\mathbf{C}^{p}}:\dot{\mathbf{C}}^{p}+\dfrac{\partial\Psi}{\partial\mathbf{Q}}\cdot\dot{\mathbf{Q}}+\dfrac{\partial\Psi}{\partial c}\dot{c}+\dfrac{\partial\Psi}{\partial c_{,\mathbf{X}}}\cdot\dot{c}_{,\mathbf{X}}+\dfrac{\partial\Psi}{\partial c_{,jj}}\dot{c}_{,ii}+\dfrac{\partial\Psi}{\partial\dot{c}}\ddot{c}\right)\mathrm{d}\Omega_{0} \\
	&\leq \int\limits_{\Omega_{0}}\left(\dfrac{1}{2}\mathbf{S}:\dot{\mathbf{C}}+\bm{\xi}\cdot\dot{c}_{,\mathbf{X}}-\pi\dot{c}\right)\mathrm{d}\Omega_{0}.
	\end{aligned}
\end{equation}
So as to compare the terms on the left- and right-hand side and thus, to involve the Laplacian of $c$, the divergence theorem is applied:
\begin{equation} \label{eq:div_theorem_Lapl_c}
	\begin{aligned}
		\int\limits_{\Omega_{0}}\dfrac{\partial\Psi}{\partial c_{,jj}}\dot{c}_{,ii}\mathrm{d}\Omega_{0} &= \int\limits_{\Omega_{0}}\left[\left(\dfrac{\partial\Psi}{\partial c_{,jj}}\dot{c}_{,i}\right)_{,i}-\left(\dfrac{\partial\Psi}{\partial c_{,jj}}\right)_{,i}\dot{c}_{,i}\right]\mathrm{d}\Omega_{0} \\
		&= \int\limits_{\partial\Omega_{0}}\dfrac{\partial\Psi}{\partial c_{,jj}}\dot{c}_{,i}N_{i}\mathrm{d}\partial\Omega_{0}-\int\limits_{\Omega_{0}}\left(\dfrac{\partial\Psi}{\partial c_{,jj}}\right)_{,i}\dot{c}_{,i}\mathrm{d}\Omega_{0}.
	\end{aligned}
\end{equation}
As a consequence, terms can be compared which leads to following new relations:
\begin{equation} \label{eq:relations_cons4}
	\begin{aligned}
		\dfrac{\partial\Psi}{\partial\dot{c}} &= 0, \\
		\dfrac{\partial\Psi}{\partial c_{,\mathbf{i}}} -\left(\dfrac{\partial\Psi}{\partial c_{,jj}}\right)_{,i} &= \xi_{,i}, \\
		\dfrac{\partial\Psi}{\partial c_{,jj}} &= 0 \quad \text{on } \partial\Omega_{0}, \\
		2\dfrac{\partial\Psi}{\partial\mathbf{C}} &= \mathbf{S}, \\
		\left(\pi+\dfrac{\partial\Psi}{\partial c}\right)\dot{c} &\leq 0, \\
		\mathbb{D}^{p} &\geq 0.
	\end{aligned}
\end{equation}
The function $\beta$ stays the same as in \eqref{eq:relation_cons_beta}. Inserting this function and the relations \eqref{eq:relations_cons4} into the microforce balance \eqref{eq:microforce_balance} leads to the equation
\begin{equation} \label{eq:balance_law_c4_ductile}
	\left(\dfrac{\partial\Psi}{\partial c_{,i}}\right)_{,i}-\left(\dfrac{\partial\Psi}{\partial c_{,jj}}\right)_{,ii}-\dfrac{\partial\Psi}{\partial c}-\beta\dot{c}+l=0.
\end{equation}
Inserting the Helmholtz free energy from \eqref{eq:Helmholtz_ductile2} for $n=4$ into \eqref{eq:balance_law_c4_ductile} leads to the governing equations for the phase-field's evolution considering the fourth-order phase-field theory:
\begin{equation} \label{eq:c4_equil_ductile}
	\left\{\begin{alignedat}{2}
		\dfrac{2l_{0}}{\mathcal{G}_{c}^{0}}\left(\beta_{e}g'\left(c\right)\mathcal{H}+\beta_{p}g_{p}'\left(c\right)\left<\Psi_{p}-\Psi_{0}\right>\right) + c - 2l_{0}^{2}\Delta c +l_{0}^{4}\Delta\left(\Delta c\right) &= 1 && \quad\text{on } \Omega_{0}\times\left(0,T\right) \\
		\Delta c &= 0 && \quad \text{on } \partial\Omega_{0}\times\left(0,T\right) \\
		\nabla\left(l_{0}^{4}\Delta c-2l_{0}^{2}c\right)\cdot\mathbf{N} &= 0 && \quad \text{on } \partial\Omega_{0}\times\left(0,T\right) \\
c &= c_{0} && \quad \text{on } \Omega_{0}\times0  
	\end{alignedat}\right.
\end{equation}
Note that the threshold $\Psi_{0}$ has been added just as for the second-order phase-field theory (see \eqref{eq:c2_equil_ductile}). The boundary conditions on the phase-field result from the homogeneous natural boundary conditions arising from the derivation of the weak form \citep{11_PF_DissBorden}.

\subsection{Weak and semidiscrete Galerkin form}
Analogous to brittle fracture, the weak form for the second-order phase-field theory for ductile fracture can be derived. Considering the critical fracture energy density $\mathcal{G}_{c}=\frac{\mathcal{G}_{c}^{0}}{J}$ in the current configuration with Jacobian $J$, the system reads:

\textbf{n=2:}$\quad$ Find $\mathbf{u}\left(t\right)\in\bm{\mathcal{S}}$ and $c\left(t\right)\in\tilde{\mathcal{S}}_{2},t\in\left[0,T\right]$ so that $\forall \mathbf{w}\in\bm{\mathcal{V}}$ and $\forall q\in\tilde{\mathcal{V}}_{2}$:
\begin{equation} \label{eq:weak_2_ductile}
\begin{aligned}
\left\{\begin{alignedat}{1}
	\left(\rho \ddot{\mathbf{u}},\mathbf{w}\right)_{\Omega} + \left(\bm{\sigma},\nabla\mathbf{w}\right)_{\Omega} &= \left(\mathbf{h},\mathbf{w}\right)_{\partial\Omega_{\mathbf{h}}} + \left(\mathbf{b},\mathbf{w}\right)_{\partial\Omega_{\mathbf{g}}} \\
	\left(\dfrac{2l_{0}}{J\mathcal{G}_{c}}\left(\beta_{e}g'\left(c\right)+\beta_{p}g_{p}'\left(c\right)\left<\Psi_{p}-\Psi_{0}\right>\right)+c,q\right)_{\Omega} & \\
	+\left(4l_{0}^{2} c_{,\mathbf{X}}, q_{,\mathbf{X}}\right)_{\Omega_{0}} &= \left(1,q\right)_{\Omega} \\
	\left(\rho\mathbf{u}\left(0\right),\mathbf{w}\right)_{\Omega} &= \left(\rho\mathbf{u}_{0},\mathbf{w}\right)_{\Omega} \\
	\left(\rho\dot{\mathbf{u}}\left(0\right),\mathbf{w}\right)_{\Omega} &= \left(\rho\dot{\mathbf{u}}_{0},\mathbf{w}\right)_{\Omega} \\
	\left(c\left(0\right),q\right)_{\Omega_{0}} &= \left(c_{0},q\right)_{\Omega_{0}}.
\end{alignedat}\right.
\end{aligned}
\end{equation}
The semidiskrete Galerkin form follows as already mentioned for the weak form considering brittle fracture (see \secref{sec:weak_Gal_form}). The linearization of $g'\left(c\right)$ is done the same way as in \eqref{eq:cubic_degr_fct_lin}. Note that derivatives with respect to the reference configuration can be mapped to derivatives within the current configuration by applying the chain rule $\left(\cdot\right)_{,\mathbf{X}}=\left(\cdot\right)_{,\mathbf{x}}\mathbf{F}$. Thus, the term $\left(4l_{0}^{2} c_{,\mathbf{X}}, q_{,\mathbf{X}}\right)_{\Omega_{0}}$ in \eqref{eq:weak_2_ductile} becomes $\left(4l_{0}^{2} \mathbf{F}^{T}c_{,\mathbf{x}}^{h}, \mathbf{F}^{T}q_{,\mathbf{x}}^{h}\right)_{\Omega}$ so that the weak as well as the semidiscrete Galerkin form is stated in the current configuration.

For the fourth-order phase-field theory, the weak form considering ductile fracture reads:

\textbf{n=4:}$\quad$ Find $\mathbf{u}\left(t\right)\in\bm{\mathcal{S}}$ and $c\left(t\right)\in\tilde{\mathcal{S}}_{2},t\in\left[0,T\right]$ so that $\forall \mathbf{w}\in\bm{\mathcal{V}}$ and $\forall q\in\tilde{\mathcal{V}}_{2}$:
\begin{equation} \label{eq:weak_4_ductile}
\begin{aligned}
\left\{\begin{alignedat}{1}
	\left(\rho \ddot{\mathbf{u}},\mathbf{w}\right)_{\Omega} + \left(\bm{\sigma},\nabla\mathbf{w}\right)_{\Omega} &= \left(\mathbf{h},\mathbf{w}\right)_{\partial\Omega_{\mathbf{h}}} + \left(\mathbf{b},\mathbf{w}\right)_{\partial\Omega_{\mathbf{g}}} \\
	\left(\dfrac{2l_{0}}{J\mathcal{G}_{c}}\left(\beta_{e}g'\left(c\right)+\beta_{p}g_{p}'\left(c\right)\left<\Psi_{p}-\Psi_{0}\right>\right)+c,q\right)_{\Omega} & \\
	+\left(2l_{0}^{2} c_{,\mathbf{X}}, q_{,\mathbf{X}}\right)_{\Omega_{0}} + \left(l_{0}^{4}\Delta^{\mathbf{X}} c,\Delta^{\mathbf{X}} q\right)_{\Omega_{0}} &= \left(1,q\right)_{\Omega} \\
	\left(\rho\mathbf{u}\left(0\right),\mathbf{w}\right)_{\Omega} &= \left(\rho\mathbf{u}_{0},\mathbf{w}\right)_{\Omega} \\
	\left(\rho\dot{\mathbf{u}}\left(0\right),\mathbf{w}\right)_{\Omega} &= \left(\rho\dot{\mathbf{u}}_{0},\mathbf{w}\right)_{\Omega} \\
	\left(c\left(0\right),q\right)_{\Omega_{0}} &= \left(c_{0},q\right)_{\Omega_{0}}
\end{alignedat}\right.
\end{aligned}
\end{equation} 
where $\Delta^{\mathbf{X}}$ refers to the Laplacian with respect to the reference configuration. Note that $\Delta^{\mathbf{X}}c=\sum_{i}c_{,ii}F_{ii}^{2}$ with $c_{,ii}$ denoting the second-order derivatives with respect to the current configuration. Thus, the weak form \eqref{eq:weak_4_ductile} and corresponding semidiscrete Galerkin form can be formulated in the current configuration.

The semidiscrete Galerkin forms for both, the second-order and fourth-order model, can be analogously derived as for brittle fracture (see \secref{sec:weak_Gal_form}).

\subsection{Residual vectors for the time integration scheme}
Analogous to brittle fracture, for the second-order phase-field theory of ductile fracture, the residual vectors based on $\eqref{eq:weak_2_ductile}_{2}$ within the generalized-$\alpha$ scheme are given by
\begin{equation} \label{eq:res_vecs_c2_ductile}
	\begin{aligned}
	\begin{alignedat}{2}
		\mathbf{R}^{c}&=\{R_{A}^{c}\}, \\
		R_{A}^{c} &= \left(1,N_{A}\right)_{\Omega} &&- \left(\dfrac{2l_{0}}{J\mathcal{G}_{c}}\left(\beta_{e}g'_{lin}\left(c^{h}\right)+\beta_{p}g_{p,lin}'\left(c^{h}\right)\left<\Psi_{p}-\Psi_{0}\right>\right)+c^{h},N_{A}\right)_{\Omega} \\
			& &&- \left(4l_{0}^{2}c^{h}_{,I},N_{A,I}\right)_{\Omega}
			\end{alignedat}
	\end{aligned}
\end{equation}
and for the corresponding fourth-order theory based on $\eqref{eq:weak_4_ductile}_{2}$ by
\begin{equation} \label{eq:res_vecs_c4_ductile}
	\begin{aligned}
	\begin{alignedat}{2}
		\mathbf{R}^{c}&=\{R_{A}^{c}\}, \\
		R_{A}^{c} &= \left(1,N_{A}\right)_{\Omega} &&- \left(\dfrac{2l_{0}}{J\mathcal{G}_{c}}\left(\beta_{e}g'_{lin}\left(c^{h}\right)+\beta_{p}g_{p,lin}'\left(c^{h}\right)\left<\Psi_{p}-\Psi_{0}\right>\right)+c^{h},N_{A}\right)_{\Omega} \\
		 & &&- \left(2l_{0}^{2}c^{h}_{,I},N_{A,I}\right)_{\Omega} - \left(l_{0}^{4}c^{h}_{,II},N_{A,JJ}\right)_{\Omega}.
	\end{alignedat}
	\end{aligned}
\end{equation}