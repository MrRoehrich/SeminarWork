\begin{abstract}
Next to sharp interface models for the description and prediction of crack growth, diffuse interface models represent a powerful method so as to model cracks and their propagation. This work shortly illustrates two methods using a sharp interface model and afterwards focuses on a phase-field description of dynamic fracture. For brittle fracture, the variational formulation of the problem is presented and for ductile fracture, the governing equations are derived with the help of balance laws. Both approaches include a phase-field approximation. Differences between these models are emphasized and the resulting coupled systems of partial differential equations are given. A fourth-order model is exposed by manipulating the crack density functional of the second-order model. In order to numerically approximate their solution, the semidiscrete Galerkin forms of the governing equations are derived. Since the fourth-order model require a higher regularity of the phase-field's solution, a Finite Element Analysis within an Isogeometric Analysis framework is outlined. A time integration scheme for the description of dynamic fracture is sketched. Finally, the major differences between sharp and diffuse interface models are examined and consequently, advantages of using phase-field models are underlined.
\end{abstract}

\begin{keyword}
%% keywords here, in the form: keyword \sep keyword
Phase-field \sep Fracture mechanics \sep Variational formulation \sep Brittle fracture \sep Ductile fracture \sep Isogeometric Analysis \sep Higher-order Models
%% MSC codes here, in the form: \MSC code \sep code
%% or \MSC[2008] code \sep code (2000 is the default)
\end{keyword}