\section{Conclusion} \label{sec:concl}
A phase-field approach for the sake of modelling dynamic fracture has been outlined in this work. A second- and a fourth-order phase-field theory as well as differences between modelling brittle and ductile fracture have been presented. The here shown derivation is based on the variational formulation of brittle fracture which can be seen as a basis for the regularization by the phase-field. Considering ductile fracture, microforce balance laws have been derived. Within the phase-field approximation, the length scale parameter $l_{0}$ has been introduced. A suitable time integration scheme and shortly, an Isogeometric Analysis framework have been described.

In general, there exist many different setups for modelling fracture with the help of phase-field methods. Since this work is established in a didactic focus, five general differences in these models are mentioned:
\begin{enumerate}
	\item quasi-static ($\ddot{\mathbf{u}}=\mathbf{0}$) $\leftrightarrow$  dynamic ($\ddot{\mathbf{u}}\neq\mathbf{0}$)
	\item $2^{nd}$-order- $\leftrightarrow$ $4^{th}$-order phase-field theory
	\item brittle fracture $\leftrightarrow$ ductile fracture
	\item\label{item:def} small deformations $\leftrightarrow$ large deformations
	\item\label{item:el_pl} elasticity $\leftrightarrow$ plasticity.
\end{enumerate}
Especially, items \ref{item:def} and \ref{item:el_pl} strongly depend on each other. Furthermore, there are several aspects which have not been mentioned in this paper. For example, adaptive mesh refinement should be considered along the crack so as to get a higher resolution of the fracture process. Within the Isogeometric Analysis framework, this can be achieved by using so called T-Splines or by applying current methods for locally refining NURBS.

So as to conclude this work, advantages of phase-field models of dynamic fracture in contrast to common sharp interface models are presented as well as differences between these approaches are mentioned. Essentially, there are two major differences. At first, the region between damaged and undamaged material is smoothed by the phase-field. As the name already reveals, in sharp interface models there is a discontinuity representing a crack. Secondly, the shown method leads to a new partial differential equation which has to be numerically solved. In contrast to that, sharp interface models require a great effort on mesh handling and refinement. This aspect already leads to the first great advantage of phase-field methods: The step from 2D to 3D is quite straightforward whereas this step requires much more expense in sharp interface models due to the fact that the mesh handling gets more complicated. Furthermore, the smooth boundary results in the fact that the crack is not a function of the geometry or the mesh, respectively. Often, in sharp interface models the crack is restricted to edges of the mesh which is not the case for the method presented here. Thus, phase-field models do not require numerical tracking of the crack so as to model crack propagation. Actually, this fact makes phase-field methods more uncomplicated. Sharp interface models often suffer in situations with complex crack topologies including branching. This problem can be overcome relative easily by applying the described diffuse interface model. Finally, \citet{02_PF_HO_brittle} have shown that the presented fourth-order phase-field theory leads to more accurate results. Since stresses are represented more accurately, the residual stress inside the crack decreases in contrast to the second-order theory. Thus, higher and more realistic crack growth rates can be reached. Additionally, as expected, convergence rates increase for the higher-order model.

All these aspects show the great potential of modelling fracture with the help of phase-fields. This work gives a first insight into this topic and the derivation of the governing equations. Of course, many topics have not been enlightened here but in the context of a didactic focus, this paper is thought to introduce and especially, motivate this method and should thus, help with a good start into the subject matter.

