\section{Introduction} \label{sec:intro}

So as to shorten development times, the prediction of fracture and failure in materials play a significant role for engineering designs. Mostly, experiments are not profitable due to high expense and enormous costs. So as to overcome these problems, computational models have been proposed in order to accurately model theses physical effects. 

% Sharp interface models
Finite Element Methods (FEM) are commonly used to numerically approximate crack behaviour in a given context. Here, a robust and efficient implementation is to the fore. A major question arising while modelling fracture processes within the FEM context is how to describe the splitting of the material. Obviously, it is not possible to simply let connected Finite Elements be separated from one point in time to the other. Thus, so called \textit{sharp interface models} have been proposed by enriching the displacement field with discontinuities or by inserting discontinuities by means of mesh handling. So as to give a short introduction to these approaches, subsequently, two of them are shortly outlined. Since these kind of strategies are not topic of this work, it is referred to \ref{appsec:sharp} where they are explained in a bit more detail.

The \textit{virtual crack closure technique} represents a crack by introducing one-dimensional discontinuities by a line of nodes. So as to let several Finite Elements separate from each other, there exist nodes with identical coordinates. In the case of undamaged material, multipoint constraints are imposed on these nodes so that they behave in the same manner. \cite{03_SotA_virtClos} 

In the \textit{cohesive segments method} the crack is represented by a set of overlapping cohesive segments. By means of these segments, displacement jumps are introduced as soon as the current stress state violates a given fracture criterion. However, this approach leads to several disadvantages: Quadrature rules have to be adapted, nodes need to have a variable number of degrees of freedom, there is a high accuracy error, the stress state at the tip of a segment is not well described and there is a problem is the specification of the geometry of the cohesive surfaces. \cite{02_SotA_cohes}\cite{01_SotA_cohes_dyn}

%\subsection{Fracture mechanics}
However, extending these formulations to three dimensions has proven to be difficult. Thus, today many approaches base on the variational formulation of brittle fracture. \citet{02_B_VarBrittle} propose that the solution of the fracture problem based on Griffith's theory is given by the minimizer of a global energy functional. In Griffith's theory a critical energy release rate determines crack nucleation and propagation. Accuracy and robustness of the variational formulation in two and three dimensions using the so called \textit{phase-field methods} which represents a diffuse interface model has been shown for instance by \citet{04_B_VarBrittleProve2} and \citet{03_B_VarBrittleProve1}. In there, a derivation of the phase-field based on continuum mechanics and thermodynamics is presented. Additionally, in \cite{04_B_VarBrittleProve2} the model distinguishing between tensile and compressive effects on crack growth has been established. Successful extensions to dynamic problems have been given by \citet{01_PF_dyn_brittle}, \citet{05_B_dynExtension1}, \citet{06_B_dynExtension2}, \citet{07_B_dynExtension3} and \citet{08_B_dynExtension4}. Actually, for example \citet{10_PH_Mode3} do not base their dynamic phase-field formulation on Griffith's theory of brittle fracture. However, this paper only deals with Griffith's theory as used for example by \citet{08_PF_Gammac2} since this theory is well understood and it has been proven to be useful in engineering applications \citep{01_PF_dyn_brittle}. The established phase-field models using the variational formulation of fracture have been widely numerically used and proved in different engineering contexts, for instance by \citet{11_B_EngProb1} and \citet{12_B_EngProb2} for hydraulic fracturing, by \citet{13_B_EngProb3} for piezoelectric ceramics, by \citet{14_B_EngProb4} for rubbery polymers and by \citet{15_B_EngProb5} for thermo-elastic solids. All these examples outline the relevance and the attractiveness of this formulation. In fact, for example \citet{03_PF_ductile} have shown that the phase-field method can also be used so as to model ductile fracture. Nevertheless, this work focuses on brittle fracture. A basic derivation of the governing equations considering ductile fracture is given in \ref{appsec:ductile}.

%\subsection{Phase-field theory}
In general, \textit{phase-field methods} are used so as to model interfaces between regions of different phases. Examples are the modelling of solidification behaviour, interfaces between fluids with different properties or, as discussed in this work, fracture. In the latter case, the interface describes the region between undamaged and totally broken material. Within the brittle fracture context, the phase-field theory can be seen as a regularization of the mentioned variational formulation. In there, the abrupt change between elastic behaviour and crack nucleation is smoothed as well as the boundary between undamaged and damaged material. Thus, no discontinuities are introduced into the body and the transition between two different phases is smoothed as it happens in a diffusive process. That is why the phase-field method belongs to the so called \textit{diffuse interface models}.

%\subsection{Isogeometric Analysis}
Isogeometric Analyis (IGA) is a powerful method in which functions used for the description of Computer Aided Design (CAD) geometries are adopted as a basis for analysis. Within the FEM context, no mesh generation is required and IGA allows for a high global regularity of the numerical solution. Especially, for $C^{2}$-continuity, standard Finite Elements quickly reach their beneficial properties. The IGA-framework has been established by \citet{09_B_IGA1}. The main concept lies in the usage of B-splines so as to model a geometry or, within the FEM context, in order to get a higher regularity using these functions as ansatz functions. The introduction of hierarchical B-Splines by \citet{18_IGA_HierBSplines} offered the possiblity of local refinement. This has then been expanded to locally refining NURBS (Non-Uniform Rational B-Splines). \citet{16_IGA_TSplines} have introduced so called T-Splines for the sake of a more efficient discretization of surfaces in contrast to hierarchical B-Splines. Local refinement strategies for these have been established as well as for NURBS, so called LR NURBS (Locally Refined NURBS) \cite{17_IGA_LRNURBS}.

% Fracture & IGA & phase-field
There exist many publications about fracture computations within an IGA framework. Just to mention two, \citet{19_B_FracIGA1} have developed an interface element based on IGA in the context of a sharp interface model of fracture and \citet{20_B_FracIGA2} have made use of an extended IGA (XIGA) framework for the sake of modelling dynamic fracture in multiphase piezoelectric/piezomagnetic composites. In \citet{11_PF_DissBorden} results of several papers concerning phase-field methods of dynamic fracture within an IGA framework are summarized. In there and in the corresponding papers \cite{03_PF_ductile}\cite{02_PF_HO_brittle}\cite{01_PF_dyn_brittle}, it has been observed that this complete framework works well for complex crack patterns and nucleation, even in three dimensions. By reason of a proposed higher order model, the used IGA framework by \citet{09_B_IGA1} has been proven to be very beneficial so as to achieve higher regularity in the numerical solution. In \cite{01_PF_dyn_brittle}, an adaptive refinement strategy using T-Splines has been proposed and numerically investigated.

% Outline of this work
In this work, firstly, the governing equations for obtaining stress equilibrium and for modelling the evolution of the phase-field are derived (\secref{sec:formul}). In there, brittle fracture is assumed and its variational formulation is outlined. Its minimizer is found by solving the Euler-Lagrange equation. At the end of \secref{sec:formul}, shortly, differences in modelling ductile fracture are pointed out. Since this works focuses on brittle fracture, a detailed derivation of the governing equationa for this type of fracture can be found in \ref{appsec:ductile}. \secref{sec:num_formul} focuses on the approximation of the solution of the governing equations including the derivation of the weak and semidiscrete Galerkin form. Shortly, a motivation for using an IGA framework so as to numerically solve these forms is given as well as a short introduction to IGA. Additionally, an appropriate time integration scheme for the given problem is given. This work finishes with \secref{sec:concl} by summarizing the phase-field model of dynamic fracture. Especially, phase-field methods of fracture are compared to sharp interface models of fracture so as to highlight the advantages of diffuse interface models for this type of problems.




