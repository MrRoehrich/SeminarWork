\section{Introduction} \label{sec:intro}

So as to shorten development times the prediction of fracture and failure in materials play a significant role for engineering designs. Mostly, experiments are not profitable due to high expense and enormous costs. So as to overcome these problems, computational models have been proposed in order to accurately model theses phyical effects. A widely used model is provided by  Griffith's theory in which a critical energy release rates determines crack nucleation and propagation.

Finite Element Methods (FEM) are commonly used to numerically approximate crack behaviour in a given context. Here, a robust and efficient implementation is to the fore. A major question arising while modelling fracture processes within the FEM context is how to describe the splitting of the material. Obviously, it is not possible to simply let connected Finite Elements be separated from one point in time to the other. Thus, so called sharp interface models have been proposed by enriching the displacement field with discontinuities or by inserting discontinuities by means of mesh handling. So as to give an idea to these methods, two of them are presented in the following. At first, the virtual crack closure technique is presented and afterwards, a cohesive segments method is outlined.

\subsection{Virtual crack closure technique} \label{sec:intro_virtClos}
Assuming a two-dimensional plane stress and plane strain model, the virtual crack closure technique represents a crack introducing one-dimensional discontinuities by a line of nodes. The total energy release rate is computed locally at the crack front. If two Finite Elements shall be split by a crack, these elements will not share the same edge and nodes anymore. Thus, new nodes or edges have to be created. So as to handle this problem, there are nodes with identical coordinates where a crack forms. If there is no crack yet, these nodes will lie over each other and a multipoint constraint is used so as to keep them at the same position. As soon as the respective Finite Elements shall be split, this constraint is disregarded and the two nodes can move apart from each other. Consequently, the elements move apart and a crack within the mesh is introduced. In order to avoid kinematically incompatible self-penetration of elements, an element-wise opening instead of a node-wise opening is required. As shown by \citet{03_SotA_virtClos}, the calculation of the strain energy release rates, which determine crack nucleation, depend on the element type. Therefore, the formulars used in the computation have to be adapted for changing shape of elements and varying number of nodes per element. Since these formulars quickly get very complicated, especially for three-dimensional problems, the implementation requires high efforts. Additionally, correction formulars for different element length and for sharp corners are required. \cite{03_SotA_virtClos} 

\subsection{Cohesive segments method} \label{sec:intro_cohes}
In the cohesive segements method the crack is represented by a set of overalpping cohesive segements. Thus, the separation process is limited to a set of discrete planes in 3D or discrete lines in 2D, respectively. As soon as a stress state violates a given fracture criterion, a new segment is created. Such a segment adds a discontinuity to an element during calculations by using the partition-of-unitiy property. For all nodes whose support is cut by a crack into two disjoint pieces or which are crossed by a cohesive segment, respectively an additional equilibrium equation is given and new degrees of freedom are added. The latter do not affect energy conservation and describe the magnitude of the displacement jump. Within the FE-approximation this means, that the approximate displacement field $\mathbf{u}^{h}=\sum_{i}\mathbf{u}_{i}N_{i}$, where $N_{i}$ denote the used shape functions, is enriched by a discontinuitiy function and asymptotic crack tip functions which are necessary if a crack does not begin on an edge. However, all this leads to several disadvantages: Quadrature rules have to be adapted, nodes need to have a variable number of degrees of fredom, there is a high accuracy error, the stress state at the tip of a segment is not well described and there is a problem is the specification of the geometry of the cohesive surfaces. \cite{02_SotA_cohes}\cite{01_SotA_cohes_dyn}

The mentioned sharp interface models are only described in order to give an introduction to the topic and especially, to later compare these methods to phase-field models of dynamic fracture. Particularly this comparison helps in emphasizing the advantages of using such a phase-field model. \hl{\text{In the following section, the formulation using the phase-field approximation is described. SECTION 3 (referenz!) focuses on its numerical formulation and ... }}

%\citet{01_B_LagrMech}.
%\citet{01_SotA_cohes_dyn} \\




