\section{Introduction} \label{sec:intro}

So as to shorten development times, the prediction of fracture and failure in materials play a significant role for engineering designs. Mostly, experiments are not profitable due to high expense and enormous costs. So as to overcome these problems, computational models have been proposed in order to accurately model theses physical effects. 

% Sharp interface models
Finite Element Methods (FEM) are commonly used to numerically approximate crack behaviour in a given context. Here, a robust and efficient implementation is to the fore. A major question arising while modelling fracture processes within the FEM context is how to describe the splitting of the material. Obviously, it is not possible to simply let connected Finite Elements be separated from one point in time to the other. Thus, so called \textit{sharp interface models} have been proposed by enriching the displacement field with discontinuities or by inserting discontinuities by means of mesh handling. So as to give a short introduction to these approaches, subsequently, two of them are shortly outlined. Since these kind of strategies are not topic of this work, it is referred to \ref{appsec:sharp} where they are explained in a bit more detail.

The \textit{virtual crack closure technique} represents a crack by introducing one-dimensional discontinuities by a line of nodes. So as to let several Finite Elements separate from each other, there exist nodes with identical coordinates. In the case of undamaged material, multipoint constraints are imposed on these nodes so that they behave in the same manner. \cite{03_SotA_virtClos} 

In the \textit{cohesive segments method} the crack is represented by a set of overlapping cohesive segments. By means of these segments, displacement jumps are introduced as soon as the current stress state violates a given fracture criterion. However, this approach leads to several disadvantages: Quadrature rules have to be adapted, nodes need to have a variable number of degrees of freedom, there is a high accuracy error, the stress state at the tip of a segment is not well described and there is a problem is the specification of the geometry of the cohesive surfaces. \cite{02_SotA_cohes}\cite{01_SotA_cohes_dyn}

% Fracture
However, extending these formulations to three dimensions has proven to be difficult. Thus, today many approaches base on the variational formulation of brittle fracture. \citet{02_B_VarBrittle} propose that the solution of the fracture problem based on Griffith's theory is given by the minimizer of a global energy functional. In Griffith's theory a critical energy release rate determines crack nucleation and propagation. Accuracy and robustness of the variational formulation in two and three dimensions using the so called \textit{phase-field methods} which represents a diffuse interface model has been shown for instance by \citet{04_B_VarBrittleProve2} and \citet{03_B_VarBrittleProve1}. In there, a derivation of the phase-field based on continuum mechanics and thermodynamics is presented. Additionally, in \cite{04_B_VarBrittleProve2} the model distinguishing between tensile and compressive effects on crack growth has been established. Successful extensions to dynamic problems have been given by \citet{01_PF_dyn_brittle}, \citet{05_B_dynExtension1}, \citet{06_B_dynExtension2}, \citet{07_B_dynExtension3} and \citet{08_B_dynExtension4}. Actually, for example \citet{10_PH_Mode3} do not base their dynamic phase-field formulation on Griffith's theory of brittle fracture. However, this paper only deals with Griffith's theory as used for example by \citet{08_PF_Gammac2} since this theory is well understood and it has been proven to be useful in engineering applications \citep{01_PF_dyn_brittle}. The established phase-field models using the variational formulation of fracture have been widely numerically used and proved in different engineering contexts, for instance by \citet{11_B_EngProb1} and \citet{12_B_EngProb2} for hydraulic fracturing, by \citet{13_B_EngProb3} for piezoelectric ceramics, by \citet{14_B_EngProb4} for rubbery polymers and by \citet{15_B_EngProb5} for thermo-elastic solids. All these examples outline the relevance and the attractiveness of this formulation. In fact, for example \citet{03_PF_ductile} have shown that the phase-field method can also be used so as to model ductile fracture. Nevertheless, this work focuses on brittle fracture. A basic derivation of the governing equations considering ductile fracture is given in \ref{appsec:ductile}.

% Phase-field
The phase-field theory can be seen as a regularization of the mentioned variational formulation. In there, the abrupt change between elastic behaviour and crack nucleation is smoothed as well as the boundary between undamaged and damaged material. Thus, no discontinuities are introduced into the body. HIER HIER

% IGA
Isogeometric Analyis (IGA) has been proven to be highly advantageous due to their high global regularity. Especially, for $C^{2}$-continuity, standard Finite Elements quickly reach their beneficial properties. The IGA-framework has been established by \citet{09_B_IGA1}. The main concept lies in the usage of B-splines so as to model a geometry or, within the FEM context, in order to get a higher regularity using these functions as ansatz functions. The introduction of hierarchical B-Splines by \citet{18_IGA_HierBSplines} offered the possiblity of local refinement. This has then been expanded to locally refining NURBS (Non-Uniform Rational B-Splines). \citet{16_IGA_TSplines} have introduced so called T-Splines for the sake of a more efficient discretization of surfaces in contrast to hierarchical B-Splines. Local refinement strategies for these have been established as well as for NURBS, so called LR NURBS (Locally Refined NURBS) \cite{17_IGA_LRNURBS}.

% Fracture & IGA & phase-field
\citep{01_PF_dyn_brittle} have made use of these concepts and combined the concepts of the variational formulation of brittle fracture, phase-field theory and an IGA framework \citet{09_B_IGA1} including local refinement. 2 WEITERE BEISPIELE





In \secref{sec:formul}, the governing equations for obtaining stress equilibrium and for modelling the evolution of the phase-field are derived. First of all, brittle fracture is assumed and its variational formulation is derived. In \secref{sec:ductile_frac} changes in the formulation considering ductile fracture are outlined. For this purpose, microforce balance laws are presented.  \secref{sec:num_formul} focuses on the approximation of the solution of the governing equations which have been derived in \secref{sec:formul}. The derivation of the weak form is shown as well as the semidiscrete Galerkin form. Shortly, an Isogeometric Analysis Framework, which is used so as to achieve higher regularity in the numerical solution, is presented. Additionally, an appropriate time integration scheme for the given problem is pointed out. This work finishes with \secref{sec:concl} by summarizing the phase-field model of dynamic fracture.




